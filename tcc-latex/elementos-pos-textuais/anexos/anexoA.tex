\anexo{Exemplo: Temperatura e Umidade}
\label{an:anexo-temperatura-umidade}

\begin{lstlisting}[
    basicstyle=\tiny,
]
#include <ESP8266WiFi.h>
#include <Adafruit_Sensor.h> 
#include <DHT.h>

DHT dht(5, DHT11);

#include <MQTT.h>

WiFiClient net;
MQTTClient mqtt;

float dadosTemperatura = 0;
float dadosUmidade = 0;

const char* ssid = "# Cipriano";
const char* senha = "senha da minha casa";

const char* token = "53027A";

String float2str(float x, byte precision = 2) {
  char tmp[50];
  dtostrf(x, 0, precision, tmp);
  return String(tmp);
}

bool conectaWiFi() {
  if (WiFi.status() != WL_CONNECTED) {
    delay(250);
    return 0;
  } else
    return 1;
}

void wdt() {
  ESP.wdtFeed();
  yield();
}

void setup() {
  Serial.begin(9600);

  WiFi.mode(WIFI_STA);
  WiFi.begin(ssid, senha);
  
  conectaWiFi();
  dht.begin();

  mqtt.begin("mqtt.rscada.ga", net);
  mqtt.connect(token, token, token);
}

unsigned long tempoTotal = 0;
  
void loop() {
  if(conectaWiFi()){

    float umidade = dht.readHumidity();
    float temperatura = dht.readTemperature();

    if (isnan(umidade) || isnan(temperatura)) {
      wdt();
      return;
    }
  
    unsigned long tempoInicial = millis();

    if(mqtt.connected()){
      mqtt.publish(String(token)+"/umidade", float2str(umidade), false, 1);
      tempoTotal = millis() - tempoInicial;

      mqtt.publish(String(token)+"/temperatura", float2str(temperatura), false, 1);

      if(tempoTotal > 0)
        mqtt.publish(String(token)+"/latencia", String(tempoTotal), false, 1);
    } else {
      mqtt.disconnect();
      mqtt.connect(token, token, token);
    }
   
    while((millis() - tempoInicial) < 200) wdt();
  }
  wdt();
}
\end{lstlisting}