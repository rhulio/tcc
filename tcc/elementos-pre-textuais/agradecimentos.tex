Ao meu pai Joaquim Francisco de Sousa pelos ensinamentos e exemplos de que, respeito e benfeitorias são proporcionais ao caráter, honestidade, responsabilidade e compromisso de um homem, bases que tornam possível o seu crescimento.
    
À minha mãe Maria Josenildes Luz Carvalho pelo dom da vida e a capacidade de ser justo em detrimento de benefícios pessoais, de poder replicar, mesmo que infinitesimalmente, a humildade e compaixão com os meus semelhantes.
    
Ao orientador José Maria Pires de Menezes Júnior por toda a evolução que obtive ao longo do curso devido seu apoio em todas as etapas de meus trabalhos, propiciando que eu obtivesse sempre os melhores resultados possíveis.
    
Às minhas irmãs e minha família, em especial aos meus avós, por serem o todo da parte que sou, demonstrando que mesmo com a distância, os laços familiares são prioridade e o fator mais importante da vida em que estamos.
    
À minha namorada Jéssica, pelo amor, companheirismo e apoio nas horas que mais precisei, por sua presença me lembrar a todo momento de que as dificuldades não passam de autoflagelações e abstrações da mente e que, o mundo torna-se pequeno quando haja a coragem necessária para vivê-lo por completo.
    
Aos amigos de infância e de todo o ensino escolar, por acreditarem na capacidade de um sonho. Por mostrarem que família independe de sangue e amizades verdadeiras podem ser perpetuadas por toda a vida, sendo base de sustentação para todos os momentos difíceis.
    
A todos os amigos que tive a honra de conhecer na graduação, que me fizeram seguir em frente e acreditar que mesmo em situações ruins, existem caminhos e possibilidades para a resolução de todas as adversidades por pior que sejam e que, por maior que seja o objetivo, ele pode ser alcançado.
    
Aos professores do Departamento de Engenharia Elétrica da Universidade Federal do Piauí que somaram de forma positiva com a minha formação e tornaram possível a compreensão da imensidão da engenharia, e também aqueles, que através de suas ações, possibilitaram a criação de uma moldura do profissional que não devo ser.