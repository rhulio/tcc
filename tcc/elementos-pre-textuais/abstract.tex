This work presents the development of a supervision and data acquisition system able to implement multiple projects in the same platform counting on visualization of its elements through a WEB interface. Communication protocols between devices and servers are selected in order to achieve greater interoperability among the most varied types of processes. It is proposed to distribute the system developed as a service, in a different way to the softwares already available in the market for the same purpose, providing in the platform the computational resources necessary to make all the modules available in the system and thus gaining advantages such as: lower cost of professional training for users, the ability to use on a variety of platforms including smartphones and tablets, use of cloud computing for the scalability of resources required for real-time application; the possibility of integration with other proprietary systems. The organization and logic of the system modules are detailed and how the information is stored. The deployed security features are discussed, as well as system access levels and other protections included. Finally, we present all the functionalities available in the developed system, named RSCADA, as well as projects of real examples to demonstrate its capacity to use.


% Separe as Keywords por ponto
\keywords{SCADA WEB. Telemetry. Internet of Things. Industry 4.0.}