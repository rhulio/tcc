Este trabalho apresenta o desenvolvimento de um sistema de supervisão e aquisição de dados capaz de implementar múltiplos projetos na mesma plataforma contando com visualização de seus elementos através de uma interface WEB. Protocolos de comunicação entre dispositivos e servidores são selecionados a fim de obter uma maior interoperabilidade entre os mais variados tipos de processos. É proposta a distribuição do sistema desenvolvido como serviço, de forma distinta aos \textit{softwares} já disponíveis no mercado com a mesma finalidade, fornecendo na própria plataforma os recursos computacionais necessários para a disponibilização de todos os módulos do sistema e com isso obtendo vantagens significativas, como: menor custo na capacitação profissional dos utilizadores, a possibilidade de utilização nas mais diversas plataformas incluindo \textit{smartphones} e \textit{tablets}, uso de computação em nuvem para a escalabilidade de recursos necessários à aplicação em tempo real e a possibilidade de integração com outros sistemas proprietários. A organização e lógica dos módulos do sistema são detalhadas e o modo de como são armazenadas as informações. Os recursos de segurança implementados são debatidos, assim como níveis de acesso do sistema e outras proteções inclusas. Por fim, são apresentadas todas as funcionalidades disponíveis no sistema desenvolvido, batizado RSCADA, assim como projetos de exemplos reais para demonstrar sua capacidade de utilização.

% Separe as palavras-chave por ponto
\palavraschave{SCADA WEB. Telemetria. Internet das Coisas. Indústria 4.0.}