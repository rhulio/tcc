\chapter{Interface de Gerenciamento: RSCADA}
\label{chap:interface-web}

A interface de gerenciamento foi desenvolvida conforme todos os conceitos descritos no Capítulo \ref{chap:sistema-proposto}, o tema utilizado para estilo da página foi desenvolvido pela empresa Colorlib com código-fonte aberto e modificado à critérios do projeto \cite{Concept}. O nome do sistema, RSCADA,  foi constituído da inicial R do nome do autor e o tipo do sistema em questão. Todo o código dos módulos foi programado utilizando a linguagem de programação PHP, com exceção da integração entre o banco de dados e o serviço VerneMQ, responsável pelo protocolo MQTT, teve por base programação LUA e os exemplos de utilização da plataforma disponíveis no Capítulo \ref{chap:resultados} que foram desenvolvidos utilizando C++ com pequenas modificações. Todas as etapas e telas do projeto são explicadas a seguir.

\section{Acesso ao Sistema}
\label{sec:acesso-sistema}
Ao acessar a interface de gerenciamento, é solicitado ao utilizador os dados de acesso, sendo eles: usuário ou e-mail e senha. Na Figura \ref{fig:figura-rscada-1}, é apresentada a tela real de acesso do sistema desenvolvido.

        \begin{figure}[!h]
		\Caption{\label{fig:figura-rscada-1} Tela de autenticação da Interface Web.}
		%\centering
		\UFCfig{}{
			\fbox{\includegraphics[width=8cm]{figuras/rscada-1.png}}
		}{
			\Fonte{O autor}
		}	
    	\end{figure}

Caso seja um novo utilizador, é fornecido um botão na tela de acesso para criação de novas contas, onde o sistema redirecionará à página de criação de contas e solicitará ao novo usuário: Nome, E-mail, Usuário e Senha, além de uma confirmação de leitura sobre os termos de serviço do RSCADA. Ao submeter o formulário de criação de nova conta, é enviado ao email do usuário, um \textit{link} contendo um código de confirmação da conta para validar o e-mail digitado. A Figura \ref{fig:figura-rscada-2} demonstra a tela de cadastro para novas contas do RSCADA.

        \begin{figure}[!h]
		\Caption{\label{fig:figura-rscada-2} Tela de cadastro para novos usuários.}
		%\centering
		\UFCfig{}{
			\fbox{\includegraphics[width=9cm]{figuras/rscada-2.png}}
		}{
			\Fonte{O autor}
		}	
    	\end{figure}

\section{Tela inicial da interface}
\label{sec:tela-inicial}
Devidamente autenticado no sistema, o usuário é redirecionado à tela inicial onde o foco são os projetos cadastrados pertencentes a ele. Informações sobre a quantidade de Clientes associados aos projetos e botões de ações, são disponíveis na página, dentre elas: Novo Projeto, Gerenciamento dos Projetos existentes e Exclusão dos mesmos. Um menu é disponibilizado na lateral esquerda da tela, com os principais atalhos para funções do sistema, como: (i) Minha Conta, onde o usuário poderá alterar detalhes como: E-mail ou Senha, (ii) Meus Projetos, quando necessário retornar à tela inicial dos projetos, (iii) Meus Clientes, para o cadastro de clientes descritos anteriormente como Operadores, que farão uso dos projetos quando desenvolvidos, (iv) Meus Domínios, que o usuário poderá cadastrar o endereço de site o qual os clientes terão acesso ao projeto desenvolvido, (v) Alarmes, para visualizar eventos e alertas de informações que tenham sido inseridas no sistema e que não correspondem aos valores ideiais de funcionamento, entre outros. Nas Figuras \ref{fig:figura-rscada-3} e \ref{fig:figura-rscada-5}, são apresentadas capturas da tela descrita.

        \begin{figure}[!h]
		\Caption{\label{fig:figura-rscada-3} Tela inicial do sistema após autenticação.}
		%\centering
		\UFCfig{}{
			\fbox{\includegraphics[width=15cm]{figuras/rscada-3.png}}
		}{
			\Fonte{O autor}
		}	
    	\end{figure}
    	
    	\begin{figure}[!h]
		\Caption{\label{fig:figura-rscada-5} Apresentação Geral de todos os projetos cadastrados.}
		%\centering
		\UFCfig{}{
			\fbox{\includegraphics[width=15cm]{figuras/rscada-5.png}}
		}{
			\Fonte{O autor}
		}	
    	\end{figure}

A plataforma também permite o ajuste automático ao tipo de tela que o utilizador tenha, é a função que abre possibilidade para que o sistema seja adaptado à qualquer tipo de plataforma. Partindo do princípio que todo dispositivo conectado à internet tenha um navegador ou forma primitiva de um, é possível sua utilização, como exemplo, na Figura \ref{fig:figura-rscada-smartphone} que é apresentada uma captura de tela quando aberta em um \textit{smartphone}.

    	\begin{figure}[!h]
		\Caption{\label{fig:figura-rscada-smartphone} Tela inicial do sistema após autenticação em um \textit{smartphone}.}
		%\centering
		\UFCfig{}{
			\fbox{\includegraphics[width=7cm]{figuras/rscada-smartphone.png}}
		}{
			\Fonte{O autor}
		}	
    	\end{figure}
    	
\section{Criação de novos projetos}
\label{sec:criacao-projetos}
Ao clicar no botão Novo Projeto, o usuário é redirecionado à uma página solicitando o nome que se deseja para ele, conforme apresentado na Figura \ref{fig:figura-rscada-4}. Ao submeter o formulário, é apresentada uma mensagem de sucesso, caso seja possível a criação dele, em seguida, encaminhado à página inicial do projeto, representado na Figura \ref{fig:figura-rscada-novo}. 

        \begin{figure}[!h]
		\Caption{\label{fig:figura-rscada-4} Página de cadastro de novos Projetos.}
		%\centering
		\UFCfig{}{
			\fbox{\includegraphics[width=15cm]{figuras/rscada-4.png}}
		}{
			\Fonte{O autor}
		}	
    	\end{figure}
    	
    	\begin{figure}[!h]
		\Caption{\label{fig:figura-rscada-novo} Página de gerenciamento de um novo projeto.}
		%\centering
		\UFCfig{}{
			\fbox{\includegraphics[width=15cm]{figuras/rscada-novo.png}}
		}{
			\Fonte{O autor}
		}	
    	\end{figure}

Nessa nova tela é apresentada uma mensagem de que não existe nenhuma variável cadastrada e há um reforço de cor no botão de criação para indicar a relevância dessa ação. Ao clicar no botão de Nova Variável, o usuário é redirecionado à uma tela ao qual poderá escolher o tipo pretendido, entre as citadas na Seção \ref{sec:tipos-variaveis}, a variável iniciada por letra e minúscula, um nome que represente uma descrição a ela e a unidade correspondente caso exista. Na Figura \ref{fig:figura-rscada-6}, é apresentada uma captura da tela de novas variáveis.

\newpage

        \begin{figure}[!h]
		\Caption{\label{fig:figura-rscada-6} Página de cadastro de novas Variáveis.}
		%\centering
		\UFCfig{}{
			\fbox{\includegraphics[width=15cm]{figuras/rscada-6.png}}
		}{
			\Fonte{O autor}
		}	
    	\end{figure}
    	
Quando submetido o formulário, é apresentada uma tela de sucesso caso a variável seja inserida no banco de dados e então, o usuário poderá gerenciar todas as variáveis já cadastradas no projeto com informação adicional sobre a quantidade de informações já inseridas no banco de dados para aquela variável específica, conforme representado na Figura \ref{fig:figura-rscada-variaveis} e coluna Registros. O cadastro de novas variáveis também pode ser feito diretamente no módulo de aquisição de dados, onde submetendo informações citando uma variável não cadastrada, o próprio módulo identificará o tipo dela e fará a submissão ao banco de dados. Este procedimento é feito para evitar a perda de informação, caso o desenvolvedor considere novas variáveis no dispositivo do processo e não tenha atualizado no sistema \gls{SCADA} ainda.

        \begin{figure}[!h]
		\Caption{\label{fig:figura-rscada-variaveis} Gerenciamento das variáveis cadastradas no projeto.}
		%\centering
		\UFCfig{}{
			\fbox{\includegraphics[width=15cm]{figuras/rscada-variaveis.png}}
		}{
			\Fonte{O autor}
		}	
    	\end{figure}

Com as variáveis já cadastradas no sistema, é oferecida a opção de criação de novos Objetos, detalhados na Seção \ref{sec:projetos}, que quando clicado o botão, o usuário é direcionado à página ilustrada pela Figura \ref{fig:figura-rscada-7}. É solicitado o tipo do objeto, entre as opções já disponíveis na data de apresentação deste trabalho:

\begin{alineascomponto}
    \item Chave Binária: disponibiliza um botão que controla uma variável binária, à qual cada clique inverte seu valor, foi desenvolvida pensando em ser utilizada na função liga/desliga de alguma ação dentro do processo que seja necessário.
    \item Botão de Ação: disponibiliza um botão que pode oferecer o envio de um valor determinado em uma variável no instante em que for solicitado. Foi desenvolvido pensando em ser utilizado em funções que não sejam sustentadas, ou seja, tenha um único acionamento com período de duração determinado.
    \item Último Valor: disponibiliza um texto fixo com o último valor enviado pelo dispositivo da variável desejada no objeto. É dada a variação em relação ao valor imediatamente anterior à ele e o tempo que se passou desde o último envio.
    \item Gráfico de Linha: disponibiliza na tela um gráfico de linhas com uma série temporal das informações enviadas à variável escolhida. 
    \item Gráfico de Área: disponibiliza na tela um gráfico de área com uma série temporal das informações enviadas à variável escolhida.
    \item Gráfico de Barras: disponibiliza na tela um gráfico de barras com uma série temporal das informações enviadas à variável escolhida.
    \item Tabela de Informações: disponibiliza na tela uma tabela contendo uma quantidade dos últimos valores das informações enviadas à variável escolhida.
    \item Tabela de Eventos:  disponibiliza na tela uma tabela contendo os registros de alarmes e alertas sobre envio de informações que não estejam em acordo com o definido para a variável escolhida, representando também os alertas visuais que serão oferecidos ao operador quando ocorridas. Na Figura \ref{fig:figura-rscada-alarmes} são apresentados exemplos dos alarmes que são emitidos em tela para o Operador.
\end{alineascomponto}

Em seguida, são solicitados: Título, Descrição e Tamanho do Objeto, que servirão para apresentação da caixa gráfica quando inserida na interface do projeto. O Tamanho é uma seleção entre: Pequeno, Médio, Grande e Gigante, sendo respectivamente relacionado à largura ocupada da tela pelo Objeto, 25\%, 50\%, 75\% e 100\%.

\newpage

        \begin{figure}[!h]
		\Caption{\label{fig:figura-rscada-7} Página de cadastro de novos Objetos.}
		%\centering
		\UFCfig{}{
			\fbox{\includegraphics[width=15cm]{figuras/rscada-7.png}}
		}{
			\Fonte{O autor}
		}	
    	\end{figure}
    	
    	\begin{figure}[!h]
		\Caption{\label{fig:figura-rscada-alarmes} Alertas do sistema quando há uma não-conformidade da informação recebida.}
		%\centering
		\UFCfig{}{
			\fbox{\includegraphics[width=15cm]{figuras/rscada-alarmes.png}}
		}{
			\Fonte{O autor}
		}	
    	\end{figure}
    	
Quando criado o objeto, é solicitada a edição dele para a determinação dos parâmetros referentes ao tipo específico escolhido, conforme apresentado na Figura \ref{fig:figura-rscada-10}. Na tela referente à edição do Objeto, é selecionada a variável que o objeto manipulará e a Janela de Tempo que seria a quantidade de pontos de informação à serem consideradas pelo sistema quando gerar o Objeto na interface de gerenciamento. Na Figura \ref{fig:figura-rscada-editar-objeto}, é apresentado um exemplo de edição de um objeto do tipo Tabela de Informações em uma Nova Variável e Janela de Tempo de 5 minutos.
        
        \begin{figure}[!h]
		\Caption{\label{fig:figura-rscada-10} Objeto recém-criado.}
		%\centering
		\UFCfig{}{
			\fbox{\includegraphics[width=12cm]{figuras/rscada-10.png}}
		}{
			\Fonte{O autor}
		}	
    	\end{figure}

    	\begin{figure}[!h]
		\Caption{\label{fig:figura-rscada-editar-objeto} Edição de objeto para determinação de parâmetros.}
		%\centering
		\UFCfig{}{
			\fbox{\includegraphics[width=15cm]{figuras/rscada-editar-objeto.png}}
		}{
			\Fonte{O autor}
		}	
    	\end{figure}

Após a organização dos Objetos e Variáveis da etapa anterior, deve ser feita a inclusão no menu Meus Clientes, dos operadores que utilizarão o projeto criado. A página Meus Clientes, disponibiliza um botão Novo Cliente, que após clicado, solicita dados do cliente, como: Nome, E-mail, Usuário e Senha que servirão para acesso do mesmo à interface de gerenciamento quando incluso no projeto. Na Figura \ref{fig:figura-rscada-novo-cliente}, é apresentada a página de inclusão de novos clientes.

        \begin{figure}[!h]
		\Caption{\label{fig:figura-rscada-novo-cliente} Inserção de novo cliente ao sistema.}
		%\centering
		\UFCfig{}{
			\fbox{\includegraphics[width=15cm]{figuras/rscada-novo-cliente.png}}
		}{
			\Fonte{O autor}
		}	
    	\end{figure}
    	
\quad

\quad
    	
A submissão do formulário com os dados do cliente, desde que corretamente inseridos no banco de dados, resultará numa página com uma mensagem de sucesso e o usuário será direcionado à tela de gerenciamento de todos os clientes cadastrados no sistema, outras ações são disponíveis também nesta seção como o gerenciamento do cliente, alteração de seus dados de cadastro e a exclusão do mesmo, conforme apresentado na Figura \ref{fig:figura-rscada-clientes}.
    	
    	\begin{figure}[!h]
		\Caption{\label{fig:figura-rscada-clientes} Gerenciamento dos clientes cadastrados no sistema.}
		%\centering
		\UFCfig{}{
			\fbox{\includegraphics[width=15cm]{figuras/rscada-clientes.png}}
		}{
			\Fonte{O autor}
		}	
    	\end{figure}
    	
Após inserido um novo cliente, é possível fazer a associação do mesmo ao projeto criado anteriormente. Na Figura \ref{fig:figura-rscada-8} são apresentados detalhes de como é realizada esta ação no sistema. Desde que seja corretamente associado o cliente ao projeto e inserida esta informação no banco de dados, é retornada uma página de sucesso e o usuário é redirecionado à página de Clientes Associados. No qual podem ser verificados o \textit{Token} de acesso para envio de informações, nível do cliente e quantidade de informações enviadas por ele. Estão disponíveis também outras ferramentas, como: Editar a associação entre cliente e projeto e, a exclusão do mesmo, conforme a Figura \ref{fig:figura-rscada-9} que traz detalhes da tela.
    	
        \begin{figure}[!h]
		\Caption{\label{fig:figura-rscada-8} Associação de novo cliente ao projeto.}
		%\centering
		\UFCfig{}{
			\fbox{\includegraphics[width=15cm]{figuras/rscada-8.png}}
		}{
			\Fonte{O autor}
		}	
    	\end{figure}

        \begin{figure}[!h]
		\Caption{\label{fig:figura-rscada-9} Visão geral dos clientes cadastrados no projeto.}
		%\centering
		\UFCfig{}{
			\fbox{\includegraphics[width=15cm]{figuras/rscada-9.png}}
		}{
			\Fonte{O autor}
		}	
    	\end{figure}
    	
\section{Comparação com o Eclipse SCADA}
\label{sec:comparacao-eclipse}

Esta Seção apresenta uma comparação de como é a criação de um novo projeto em um dos \textit{softwares} disponíveis no mercado, como os citados anteriormente. O trabalho de  \citeonline{Stenio}, foi desenvolvido no \textit{software} Elipse SCADA (Seção \ref{sec:elipse}), para o monitoramento e controle de um motor trifásico de indução. Utilizando dispositivos como: (i) \gls{CLP} da fabricante WEG, modelo TPW-03-40R, (ii) módulo de expansão WEG TPW03-8AD, (iii) módulo de expansão WEG TPW03-2DA, (iv) inversor de frequência WEG CFW-11, (v) motor trifásico de indução e (vi) fonte de bancada, o autor implementa acionamentos, como: aceleração, desaceleração, controle de velocidade e outros, no motor de indução trifásico através do inversor comandado pelo \gls{CLP}.

Para implementação do projeto, o autor utiliza uma rede Modbus (Seção \ref{sec:modbus-serial}) para a comunicação entre os dispositivos e o sistema \gls{SCADA}, configurados através de driver e parâmetros de configuração fornecidos pela Modicon Inc. Após a comunicação concluída, são criadas no projeto, \textit{tags} de entrada e saída, similarmente às variáveis do sistema proposto. Na Figura \ref{fig:figura-stenio-geral} é apresentada a tela do módulo \textit{Organizer} ao qual o autor parametriza todas as \textit{tags}, divididas em dois grupos: Entradas e Saídas, a serem adquiridas ou disponibilizadas através do rede Modbus.

        \begin{figure}[!h]
		\Caption{\label{fig:figura-stenio-geral} Parametrização das tags.}
		%\centering
		\UFCfig{}{
			\fbox{\includegraphics[width=11cm]{figuras/stenio-geral.png}}
		}{
			\Fonte{\cite{Stenio}}
		}	
    	\end{figure}
    	
No primeiro grupo, são adicionadas as variáveis de entrada relacionadas ao acionamento, enquanto o segundo, são as saídas diretas para as bobinas de contato. Os espaços N1, N2, N3 e N4 são configurados da seguinte forma: (i) N1: endereço do escravo, (ii) N2: operando relativo ao tag, (iii) N3: não é utilizado, permanecendo valor nulo e (iv) N4: o endereço Modbus dos registradores de entrada, bobinas ou conversão analógico-digital.
\newpage
Outras funções são disponibilizadas para configuração, como: Alarmes, Relatórios e outras, mas que não são abordadas no trabalho em questão. Os endereços Modbus para configuração do item N4 são disponíveis pela tabela de endereços do equipamento, que quando digitada a variável é retornada o endereço desejado. Na Figura \ref{fig:figura-stenio-endereco-modbus} é representado este procedimento.
    	
        \begin{figure}[!h]
		\Caption{\label{fig:figura-stenio-endereco-modbus} Endereços Modbus dos dispositivos utilizados.}
		%\centering
		\UFCfig{}{
			\fbox{\includegraphics[width=9cm]{figuras/stenio-endereco-modbus.png}}
		}{
			\Fonte{\cite{Stenio}}
		}	
    	\end{figure}

 Após configuradas todas as \textit{tags} com seus respectivos endereços, é possível a organização da \gls{IHM} disponível no \textit{software}. Em seu trabalho, o autor utiliza 3 pares de botões de ação, sendo eles: (i) habilita geral, (ii) sentido de giro e (iii) gira/pára, também, 3 telas de informações, sendo elas: (i) tensão da fonte de bancada, (ii) velocidade do motor e a (iii) situação do eixo. Na Figura \ref{fig:figura-stenio-tela-inicial}, é apresentada a \gls{IHM} finalizada do trabalho.
    	
        \begin{figure}[!h]
		\Caption{\label{fig:figura-stenio-tela-inicial} Tela de supervisão do processo.}
		%\centering
		\UFCfig{}{
			\fbox{\includegraphics[width=15cm]{figuras/stenio-tela-inicial.png}}
		}{
			\Fonte{\cite{Stenio}}
		}	
    	\end{figure}
    	
        \begin{figure}[!h]
		\Caption{\label{fig:figura-stenio-velocidade} Curva de Velocidade vs Tensão.}
		%\centering
		\UFCfig{}{
			\fbox{\includegraphics[width=13cm]{figuras/stenio-velocidade.png}}
		}{
			\Fonte{\cite{Stenio}}
		}	
    	\end{figure}
    
    	
Através do sistema \gls{SCADA}, \citeonline{Stenio} registra diferentes dinâmicas do processo montado. Como na Figura \ref{fig:figura-stenio-velocidade}, em que a velocidade do motor é variada de acordo com a entrada fornecida pela fonte de bancada ao conversor analógico-digital do \gls{CLP}. A curva obtida tem comportamento linear, atingindo a velocidade máxima do motor,  1800 rpm, quando atingida a tensão máxima de referência da fonte, 10 volts. Outras observações feitas pelo autor são o tempo de aceleração e desaceleração do motor, que obtém valores próximos a 12 segundos.

O processo de criação de um projeto nesta seção é similar ao RSCADA, são configuradas as variáveis utilizadas pelo processo em que, também dispõem de diferentes tipos, assim como diversas formas de representação gráfica das informações disponíveis ou botões de ações. A forma de aquisição de dados é distinta entre eles, no \textit{software} Elipse \gls{SCADA} existe uma conexão física entre o dispositivo do qual são adquiridas as informações ou atuado e, o servidor onde é executado o sistema. É utilizado um \textit{driver} de comunicação para obtenção direta dos dados no dispositivo, enquanto que no RSCADA, os dados são adquiridos essencialmente através da internet, necessitando que os dispositivos conectados suportem pelo um dos dois protocolos utilizados, ou, haja um intermediário capaz de servir como \textit{driver} de comunicação entre o dispositivo e o servidor. Apesar disso, a configuração do sistema em relação à aquisição de dados é similar, no primeiro, é realizada através dos endereços Modbus, configurados em cada variável, enquanto que no segundo, com o uso de um único \textit{token} e o conceito de chave/valor, é possível o envio das informações de forma intuitiva apenas através de um ou mais nomes de variável. Apesar de terem em comum funcionalidades nativas de um sistema \gls{SCADA}, o RSCADA tem a vantagem de suportar múltiplos usuários e projetos, de forma isolada, utilizando a mesma estrutura de servidores, já que é concebido em nuvem. Enquanto o primeiro \textit{software} é necessária uma instalação para cada processo utilizado, apenas em um sistema operacional compatível ou possíveis complementos demandados por ele, o RSCADA é disponibilizado através do navegador já em sua forma final de uso, sendo necessárias apenas pequenas configurações como a criação de um novo projeto para iniciar o uso.
    	
\section{Síntese}
\label{sec:sintese-rscada}

A abordagem utilizada neste Capítulo, pode ser sintetizada por um Diagrama de Caso de Uso, descrevendo a sequência e as unidades de interação com o sistema, disponível na Figura \ref{fig:figura-diagrama-uso}. Foi comparado um \textit{software} disponível no mercado com proposta parecida, suas vantagens e desvantagens em relação à ele. Tendo sido detalhadas todas as funcionalidades propostas desenvolvidas, o capítulo seguinte traz uma aproximação do uso real deste sistema, com base em exemplos de diversas situações de utilização do mesmo. Os dois protocolos de aquisição de dados, \gls{MQTT} e \gls{HTTP}, são utilizados para demonstrar a facilidade de uso e a capacidade de integração do sistema.

        \begin{figure}[!h]
		\Caption{\label{fig:figura-diagrama-uso} Diagrama de caso de uso sintetizando os níveis de permissão do sistema.}
		%\centering
		\UFCfig{}{
			\fbox{\includegraphics[width=15cm]{figuras/figura-diagrama-uso.pdf}}
		}{
			\Fonte{O autor}
		}	
    	\end{figure}
    	