\chapter{Conclusões e Trabalhos Futuros}
\label{chap:conclusoes-e-trabalhos-futuros}

No Capítulo \ref{chap:sistema-proposto}, foi proposto o desenvolvimento de um sistema \gls{SCADA}, seguindo o modelo de distribuição \gls{SaaS}, fornecendo todos os elementos necessários para seu funcionamento. As vantagens e desvantagens do sistema proposto foram discutidas, assim como possíveis soluções para amenização destes problemas. A estrutura lógica base para o desenvolvimento foi detalhada, como: a hierarquia, tipos de variáveis, formas de aquisição e  modelo para armazenamento dos dados. Implementações de segurança foram apresentadas, como: criptografia, proteções contra tentativas de acesso mal intencionadas e níveis de controle de acesso dos usuários ao sistema. Por fim, foram apresentadas formas de utilização e consumo de serviços possíveis na plataforma.

Através deste modelo proposto, os módulos base para o funcionamento do sistema foram fielmente desenvolvidos. No Capítulo \ref{chap:interface-web}, foi introduzido o módulo da Interface de Gerenciamento do sistema, já batizado de RSCADA. Foram apresentados os serviços responsáveis pela disponibilidade da plataforma, assim como as linguagens utilizadas para a programação dela. As  etapas de uso do sistema desenvolvido foram demonstradas, como: (i) Autenticação ao Sistema, (ii) Cadastro de novos usuários, (iii) Criação de novos projetos e inclusões de objetos, variáveis e clientes à eles, (iv) Alarmes e notificações disponíveis para não-conformidades e (v) Ferramentas de Gerenciamento dos elementos já inseridos no sistema. Por fim, apresentada uma síntese das  funcionalidades do sistema com base na hierarquia dos usuários.

No Capítulo \ref{chap:resultados}, foram desenvolvidos quatro exemplos de projetos possíveis à serem implementados no sistema RSCADA, baseando-se em condições e necessidades de usos reais distintos. Os dois primeiros, utilizando microcontroladores diferentes e o protocolo \gls{MQTT}, ofereceram uma ideia sobre o uso na prática de pequenos dispositivos ou a integração direta de outros, como os \glspl{CLP}, que através deste protocolo podem realizar comunicação direta com o sistema \gls{SCADA}. Foram coletados dados, como: Temperatura, Umidade, Nível de Sinal e Latência de Conexão com total de quase 22 milhões de registros de informações nestes dois exemplos em poucos dias de captura, demonstrando a estabilidade e a capacidade computacional do sistema. Os dois últimos exemplos, possibilitaram a demonstração da capacidade de integração do sistema RSCADA com outros sistemas já existentes, onde nestes exemplos, foram integrados o \textit{software} MATLAB como intermediário de um processo e um conjunto de roteadores que havia um sistema proprietário.

A aplicação do sistema desenvolvido na prática, atendeu à todas as premissas iniciais sobre a facilidade de utilização e isenção de qualificação no gerenciamento de servidores que antes seria necessário. O principal potencial deste modelo é a facilidade de implementação de um novo projeto partindo do zero, sendo possível iniciar a aquisição de dados com novas plantas em minutos. Trabalhos Científicos focados na aquisição de dados em uma maior escala ou de forma remota e que, antes dependiam de conhecimento técnico acerca de servidores ou armazenamento destes dados, agora podem ser feitos em poucos cliques, apenas com um pequeno tratamento no envio destas informações à esta plataforma.

Apesar das desvantagens discutidas ao longo do trabalho sobre o sistema proposto, a gama de aplicações que se obtém um benefício pela implementação deste serviço em nuvem, é superior aos processos que tenham prejuízo pelo pequeno atraso no transporte das informações. Desta forma, deve ser feita uma ponderação no instante da escolha sobre qual sistema \gls{SCADA} será mais adequado ao processo estudado, sobre o quão necessário se faz a ausência do tempo de atraso ou a faixa de segurança que este tempo não causaria pertubações, além das implicações que a ausência de conexão possam causar ao processo.

\section{Plataforma Estudantil}
\label{sec:plataforma-estudantil}

Este trabalho tem como um de seus propósitos primordiais, a potencialização de trabalhos científicos desenvolvidos em meio acadêmico que, por muitas vezes houvessem um distanciamento entre o assunto trabalhado e a exigência de capacitação para o desenvolvimento de uma plataforma online, que faça coleta e análise de informações de várias grandes áreas da Engenharia Elétrica. Ou também, casos em que haja a necessidade de estudos de plantas industriais ou microcontroladores com sistemas \gls{SCADA}. Para isto, são disponibilizadas ferramentas completas se comparado ao uso de \textit{softwares} comerciais, de forma gratuita para estes estudantes. Com a plataforma online já desenvolvida e pronta para uso, é otimizado o desenvolvimento dos projetos que terão foco apenas no uso da integração do RSCADA.

Com o objetivo de ouvir e atender pedidos dos usuários sobre o desenvolvimento de novas funcionalidades ou a manutenção das já existentes, ferramentas de colaboração são integradas à interface de gerenciamento. Desta forma, é possível melhorar continuamente a qualidade de interação dos estudantes com a plataforma e possibilita que os resultados sejam atingidos de forma mais rápida.

\section{Trabalhos Futuros}
\label{sec:trabalhos-futuros}

Num contexto geral, a ideia de um sistema \gls{SCADA} estar inserido em um processo, é basicamente, sua execução em servidores locais com uma \gls{IHM} associada à estes. Quando se propõe a execução desta interface de gerenciamento diretamente pela \gls{WEB}, outros dispositivos úteis com maior mobilidade são inseridos, como: \textit{smartphones} e \textit{tablets}. Apesar da conexão à internet nestes dispositivos serem substancialmente através de navegação \gls{WEB}, outras funcionalidades do sistema operacional poderiam ser aproveitadas pelo sistema desenvolvido neste trabalho caso houvesse um aplicativo nativo para eles. Estes dispositivos poderiam por exemplo, servir como um intermediário na comunicação com sensores que trabalhem com \textit{bluetooth}, ou, auxiliar a configuração inicial de dispositivos projetados para trabalhar com redes Wi-Fi, possibilitando por exemplo, que o RSCADA pudesse identificar novos dispositivos na rede e gerar todos os parâmetros necessários para o funcionamento imediato.

Além do desenvolvimento destes aplicativos para plataformas nativas, um possibilidade é o projeto de placas eletrônicas, que serviriam como \textit{drivers} de comunicação entre o RSCADA e dispositivos do processo que estejam legados ou que tenham difícil integração com os protocolos utilizados. Desta forma, haveria um ganho de retrocompatibilidade e a possibilidade de execução de algumas tarefas locais quando necessárias.

Por fim, outra possibilidade é o desenvolvimento de um módulo capaz de atender toda a plataforma de forma local que, na presença de conexão à internet, seja sincronizada com a estrutura em nuvem. Isto possibilitaria a mitigação do problema de atraso em transporte, já que se reduziria este tempo para algo em torno de 2 milisegundos e funcionaria como um intermediário para o pré tratamento dos dados.