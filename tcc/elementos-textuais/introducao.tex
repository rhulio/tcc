\chapter{Introdução}
\label{chap:introducao}

A indústria busca frequentemente métodos que possam agilizar e aumentar a eficiência nos seus processos, alguns trabalhos implementam automação para isto, como o \cite{IndustriaEficiencia}, que em um sistema de abastecimento de água, consegue a diminuição no desgaste de motores, menor consumo de energia e um maior controle do processo com supervisão em tempo real através de uma análise detalhada sobre ele. Para que isso aconteça, fazem-se necessárias ações e ferramentas específicas para conduzir uma mudança de forma significativa.
Uma gestão interna, confiável e integrada, aumenta a produtividade e diminui o tempo de atuação em  determinadas tarefas, \cite{InterfaceTempoReal} demonstra que com o auxílio de uma Interface em Tempo Real para o controle de um equipamento de produção, é possível o aumento da produtividade além de prever a capacidade produtiva distribuindo eficientemente os recursos de produção. Existe uma infinidade de fatores à serem considerados para a manutenção de um processo, tais como: materiais, equipamentos e qualificação de colaboradores através de procedimentos que sejam capazes de oferecer autonomia e continuidade no serviço. 

Com o surgimento de computadores, conectividade e a inclusão de máquinas automáticas no ambiente de trabalho, mediante a terceira revolução industrial, iniciou-se uma necessidade de controle de todas as etapas do processo produtivo, não somente sobre a atuação dos profissionais, mas também sobre as informações específicas de partes do processo. Tornaram-se indispensáveis nestes casos, o uso de tecnologias para que a possibilidade de tomada de decisões, que antes não seriam possíveis devido uma infinidade de informações terem que ser analisadas de forma manual, ou simplesmente não poderem ser adquiridas, sejam agora facilmente implementadas. Uma indústria em que todas as partes do processo são conectadas à rede, denominada "Indústria 4.0", implica em uma quarta revolução industrial, em que seu histórico, desde a evolução da máquina à vapor ao uso de motores movidos à eletricidade e, em seguida, o uso da eletricidade para automatização do processo produtivo através da eletrônica, dá um passo mais adiante agora, a coleta extensiva destas informações que abre possibilidades para manutenções de diagnóstico e/ou preditivas que evitariam eventuais paradas, ou outras tarefas que resultem em ineficiência nas tarefas humanas ou em uso excessivo de recursos.

Comunicação hoje, é algo de vital importância e, com a transição para a Indústria 4.0, serão necessários mecanismos que possam gerir e compatibilizar toda a informação, recursos computacionais que façam o processamento e canais que disponibilizem de forma remota estes dados.

Debater sobre este problema citando um caso que essa dificuldade causou, algum acidente, algum desperdício de dinheiro, ineficiência, dados concretos.

Existem muitos dados, sensores, processos mais complexos a cada dia, dificuldade de mudança ou upgrade no hardware local para acompanhar isso e mostrar porque o processamento remoto é a solução.

\section{Objetivos}
\label{sec:objetivos}

\subsection{Objetivo Geral}
\label{sec:objetivo-geral}

O objetivo geral deste trabalho é o desenvolvimento de um sistema capaz de adquirir dados provenientes de processos estudados e interagir com eles, tratá-los e armazená-los em uma estrutura com alta estabilidade e confiabilidade, oferecendo todos os recursos necessários para a utilização do mesmo, sem ser necessárias quaisquer configurações avançadas em servidores ou outros serviços como em  \textit{softwares} disponíveis no mercado.

\subsection{Objetivos Específicos}
\label{sec:objetivos-especificos}

Outros objetivos secundários podem ser alcançados através deste trabalho, como:

\begin{alineascomponto}
    \item oi
\end{alineascomponto}

\section{Organização do Trabalho}
\label{sec:organizacao-trabalho}

O restante deste trabalho está dividido conforme exposto a seguir. O Capítulo 2 é destinado às definições de dispositivos e protocolos que serão utilizados para desenvolver a ideia do sistema............