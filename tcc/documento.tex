
%%%%%%%%%%%%%%%%%%%%%%%%%%%%%%%%%%%%%%%%%%%%%%%%%%%%%%%
%% Você dever criar uma conta no ShareLatex. Depois, %%
%% vá nas opções no canto esquerdo superior da tela  %%
%% e clique em "Copiar Projeto". Dê um novo nome pa- %%
%% ra o projeto.                                     %%
%%                                                   %%
%% Os principais desenvolvedores deste template são: %%
%%                                                   %%
%%            Ednardo Moreira Rodrigues              %%
%%     (Doutorando em Engenharia Elétrica - UFC)     %%
%%                      &                            %%
%%            Alan Batista de Oliveira               %%
%%     (Graduando em Engenharia Elétrica - UFC)      %%
%%                                                   %%
%% Revisão:                                          %%
%%                                                   %%
%% - Eliene Maria Vieira de Moura;                   %%
%% - Francisco Edvander Pires Santos;                %%
%% - Izabel Lima dos Santos;                         %%
%% - Juliana Soares Lima;                            %%
%% - Kalline Yasmin Soares Feitosa.                  %%
%%                                                   %%
%% Grande parte do trabalho foi adaptado do template %%
%% da UECE elaborado por:                            %%
%% Thiago Nascimento  (UECE)                         %%
%% Project available on:                             %%
%% https://github.com/thiagodnf/uecetex2             %%
%%                                                   %%
%% "Dúvidas, esclarecimentos ou sugestões podem ser  %%
%% enviadas para o e-mail da Comissão de Serviços da %%
%% Biblioteca Universitária: csbu@ufc.br."           %%
%%                                                   %%
%%%%%%%%%%%%%%%%%%%%%%%%%%%%%%%%%%%%%%%%%%%%%%%%%%%%%%%                                  %%                                                   %%
%% Alterado em 2017 para ser base dos trabalhos de   %%
%% conclusão de curso da Enegenahria Elétrica        %%
%% da UFPI.                                          %%
%% Prof. José Maria Pires de Menezes Jr.             %%
%%                                                   %%
%%%%%%%%%%%%%%%%%%%%%%%%%%%%%%%%%%%%%%%%%%%%%%%%%%%%%%%

\documentclass[        
    a4paper,          % Tamanho da folha A4
    12pt,             % Tamanho da fonte 12pt
    chapter=TITLE,    % Todos os capitulos devem ter caixa alta
    section=Title,    % Todas as secoes devem ter caixa alta somente na primeira letra
    subsection=Title, % Todas as subsecoes devem ter caixa alta somente na primeira letra
    oneside,          % Usada para impressao em apenas uma face do papel
    english,          % Hifenizacoes em ingles
    spanish,          % Hifenizacoes em espanhol
    brazil,           % Ultimo idioma eh o idioma padrao do documento
    fleqn             % Coloca as equações alinhadas a esquerda
]{abntex2}

\input{lib/preambulo}

\usepackage{upquote}
\usepackage{listings}
\lstset{
  language=C++,
  basicstyle=\ttfamily\small, 
  keywordstyle=\color{blue}, 
  commentstyle=\color{red}, 
  extendedchars=true, 
  showspaces=false, 
  showstringspaces=false, 
  numbers=left,
  numberstyle=\tiny,
  breaklines=true, 
  breakautoindent=true, 
  captionpos=b,
  xleftmargin=0pt,
}

%%%%%%%%%%%%%%%%%%%%%%%%%%%%%%%%%%%%%%%%%%%%%%%%%%%%%
%%          Configuracoes do ufctex                %%
%%%%%%%%%%%%%%%%%%%%%%%%%%%%%%%%%%%%%%%%%%%%%%%%%%%%%

% Opcoes disponiveis

\trabalhoacademico{tccgraduacao}
%\trabalhoacademico{tccespecializacao}
%\trabalhoacademico{dissertacao}
%\trabalhoacademico{tese}

% Define se o trabalho eh uma qualificacao
% Coloque 'nao' para versao final do trabalho

\ehqualificacao{nao}

% Remove as bordas vermelhas e verdes do PDF gerado
% Coloque 'sim' pare remover

\removerbordasdohyperlink{sim} 

% Adiciona a cor Azul a todos os hyperlinks

\cordohyperlink{nao}

%%%%%%%%%%%%%%%%%%%%%%%%%%%%%%%%%%%%%%%%%%%%%%%%%%%%%
%%          Informação sobre a IES                 %%
%%%%%%%%%%%%%%%%%%%%%%%%%%%%%%%%%%%%%%%%%%%%%%%%%%%%%

\ies{Universidade Federal do Piauí}
\iessigla{UFPI}
\centro{Centro de Tecnologia}

%%%%%%%%%%%%%%%%%%%%%%%%%%%%%%%%%%%%%%%%%%%%%%%%%%%%%
%%        Informação para TCC de Graduacao %%
%%%%%%%%%%%%%%%%%%%%%%%%%%%%%%%%%%%%%%%%%%%%%%%%%%%%%

\graduacaoem{Engenharia Elétrica}
\habilitacao{bacharel} % Pode colocar tambem 'licenciada'

%%%%%%%%%%%%%%%%%%%%%%%%%%%%%%%%%%%%%%%%%%%%%%%%%%%%%
%%     Informação para TCC de Especializacao       %%
%%%%%%%%%%%%%%%%%%%%%%%%%%%%%%%%%%%%%%%%%%%%%%%%%%%%%

\especializacaoem{Descargas Atmosféricas}

%%%%%%%%%%%%%%%%%%%%%%%%%%%%%%%%%%%%%%%%%%%%%%%%%%%%%
%%         Informação para Dissertacao             %%
%%%%%%%%%%%%%%%%%%%%%%%%%%%%%%%%%%%%%%%%%%%%%%%%%%%%%

\programamestrado{Programa de Pós-Graduação em Xxxxxxx}
\nomedomestrado{Mestrado Acadêmico em Xxxxxxx}
\mestreem{Engenharia Xxxxxx}
\areadeconcentracaomestrado{Engenharia Xxxxxx}

%%%%%%%%%%%%%%%%%%%%%%%%%%%%%%%%%%%%%%%%%%%%%%%%%%%%%
%%               Informação para Tese              %%
%%%%%%%%%%%%%%%%%%%%%%%%%%%%%%%%%%%%%%%%%%%%%%%%%%%%%

\programadoutorado{Programa de Pós-Graduação em Xxxxxx}
\nomedodoutorado{Doutorado em Xxxxxxx}
\doutorem{Engenharia Xxxxxx}
\areadeconcentracaodoutorado{Engenharia Xxxxxxx}

%%%%%%%%%%%%%%%%%%%%%%%%%%%%%%%%%%%%%%%%%%%%%%
%%  Informacoes relacionadas ao trabalho     %%
%%%%%%%%%%%%%%%%%%%%%%%%%%%%%%%%%%%%%%%%%%%%%%

\autor{Rhúlio Victor Luz Carvalho Sousa}
\titulo{Desenvolvimento de um Sistema de Supervisão e Aquisição de Dados para múltiplos projetos com visualização WEB}
\data{2019}
\local{Teresina}

% Exemplo: \dataaprovacao{01 de Janeiro de 2012}
\dataaprovacao{18 de Junho de 2019}

%%%%%%%%%%%%%%%%%%%%%%%%%%%%%%%%%%%%%%%%%%%%%
%%     Informação sobre o Orientador       %%
%%%%%%%%%%%%%%%%%%%%%%%%%%%%%%%%%%%%%%%%%%%%%

\orientador{Prof. Dr. José Maria Pires de Menezes Júnior}
\orientadories{Universidade Federal do Piauí (UFPI)}
\orientadorcentro{Centro de Tecnologia (CT)}
\orientadorfeminino{nao} % Coloque 'sim' se for do sexo feminino

%%%%%%%%%%%%%%%%%%%%%%%%%%%%%%%%%%%%%%%%%%%%%
%%      Informação sobre o Co-orientador   %%
%%%%%%%%%%%%%%%%%%%%%%%%%%%%%%%%%%%%%%%%%%%%%

% Deixe o nome do coorientador em branco para remover do documento

\coorientador{Prof. Dr. Otacílio da Mota Almeida}
\coorientadories{Universidade Federal do Piauí (UFPI)}
\coorientadorcentro{Centro de Tecnologia (CT)}
\coorientadorfeminino{nao} % Coloque 'sim' se for do sexo feminino

%%%%%%%%%%%%%%%%%%%%%%%%%%%%%%%%%%%%%%%%%%%%%
%%      Informação sobre a banca           %%
%%%%%%%%%%%%%%%%%%%%%%%%%%%%%%%%%%%%%%%%%%%%%

% Atenção! Deixe o nome do membro da banca para remover da folha de aprovacao

\membrodabancadois{Prof. Dr. Luís Gustavo Mota Souza}
\membrodabancadoiscentro{Centro de Tecnologia (CT)}
\membrodabancadoisies{Universidade Federal do Piauí (UFPI)}

\membrodabancatres{Prof. Esp. Ronnyel Carlos Cunha Silva}
\membrodabancatrescentro{Centro de Tecnologia (CT)}
\membrodabancatresies{Instituto Federal do Maranhão (IFMA)}

\begin{document}	

	% Elementos pré-textuais
	\imprimircapa
	\imprimirfolhaderosto{}
	\imprimirfichacatalografica{elementos-pre-textuais/ficha-catalografica}
	%\imprimirerrata{elementos-pre-textuais/errata}
	\imprimirfolhadeaprovacao
	\imprimirdedicatoria{elementos-pre-textuais/dedicatoria}
	\imprimiragradecimentos{elementos-pre-textuais/agradecimentos}
	\imprimirepigrafe{elementos-pre-textuais/epigrafe}
	\imprimirresumo{elementos-pre-textuais/resumo}
	\imprimirabstract{elementos-pre-textuais/abstract}
	\renewcommand*\listfigurename{Lista de Figuras}
	\imprimirlistadeilustracoes
	\imprimirlistadetabelas
	%\imprimirlistadequadros
	%\imprimirlistadealgoritmos
	%\imprimirlistadecodigosfonte
	\imprimirlistadeabreviaturasesiglas
	%\imprimirlistadesimbolos{elementos-pre-textuais/lista-de-simbolos}   
	\imprimirsumario
	
	%Elementos textuais
	\textual
	\chapter{Introdução}
\label{chap:introducao}

Com o surgimento de computadores, conectividade e a inclusão de máquinas automáticas no ambiente de trabalho, mediante a terceira revolução industrial, iniciou-se uma necessidade de controle de todas as etapas do processo produtivo, não somente sobre a atuação dos profissionais, mas também sobre as informações específicas de partes do processo. Tornaram-se indispensáveis nestes casos, o uso de tecnologias para que a possibilidade de tomada de decisões, que antes não seriam possíveis devido uma infinidade de informações terem que ser analisadas de forma manual, ou simplesmente não pudessem ser adquiridas, sejam facilmente implementadas.

Uma indústria em que todas as partes do processo são conectadas à rede, denominada "Indústria 4.0", implica em uma quarta revolução industrial, em que seu histórico, desde a evolução da máquina à vapor ao uso de motores movidos à eletricidade e, em seguida, o uso da eletricidade para automatização do processo produtivo através da eletrônica, dá um passo adiante: a coleta extensiva destas informações. Isto abre possibilidades para manutenções de diagnóstico e/ou preditivas, que evitariam eventuais paradas ou outras tarefas que resultem em ineficiência nas tarefas humanas, ou, em uso excessivo de recursos.

Existem vários fatores à serem considerados para a manutenção de um processo, tais como: materiais, equipamentos e qualificação de colaboradores, através de procedimentos que sejam capazes de oferecer autonomia e continuidade no serviço. \citeonline{marcorin2003analise}, fazem uma análise sobre o quão distante pode ser o custo de um processo produtivo que utiliza manutenção preditiva em detrimento de manutenções corretivas e/ou preventivas. Manutenções corretivas, ao qual só serão efetuadas se houver a indisponibilidade do equipamento por quebra de peças, tornam a substituição imprevisível, ocorrendo portanto custos relacionados à parada do processo produtivo. As manutenções preventivas são normalmente agendadas e definidas de acordo com o fabricante, onde os principais componentes que sofrem desgaste, são logo substituídos em um tempo pré-determinado. Entretanto, outras variáveis podem impactar neste desgaste, perdendo sua uniformidade e, por consequência, sendo substituídas peças abaixo do tempo de seu vida. Ou no caso de falha antecipada, trazem novamente imprevisibilidade ao processo. Por fim, o autor detalha a Manutenção Preditiva, que resultará no menor custo ao processo, onde com o acompanhamento de suas informações, podem ser feitos diagnósticos que permitem o agendamento da compra de peças e intervenção para substituição delas, reduzindo a imprevibilidade de paradas do processo e eventuais custos com estoque.

São estudados métodos que possam agilizar e aumentar a eficiência nos processos. Isto pode ser verificado, por exemplo, no trabalho de \citeonline{IndustriaEficiencia}, que aplicado no sistema de abastecimento de água, conseguiu a diminuição no desgaste de motores, menor consumo de energia e um maior controle do processo com supervisão em tempo real através de uma análise detalhada sobre ele. Para que isso aconteça, fazem-se necessárias ações e ferramentas específicas para conduzir uma mudança de forma significativa.
Uma gestão interna, confiável e integrada, aumenta a produtividade e diminui o tempo de atuação em  determinadas tarefas.  \citeonline{InterfaceTempoReal} demonstra que com o auxílio de uma Interface em Tempo Real para o controle de um equipamento de produção, é possível o aumento da produtividade além de prever a capacidade produtiva, distribuindo eficientemente os recursos de produção.

Comunicação hoje, é algo de vital importância e, com a transição para a Indústria 4.0, serão necessários mecanismos que possam gerir e compatibilizar todas estas informações, advindas de uma maior quantidade de sensores e processos cada vez mais complexos. Com um maior fluxo, um grande poder em recursos computacionais fazem-se necessários para o tratamento destes dados, que dependendo do quão grande seja, é necessária a atualização do \textit{hardware} local ou a migração da estrutura para um centro de dados que os comporte. \citeonline{santos2016internet}, discute arquiteturas e tecnologias básicas para utilização do conceito de internet das coisas que buscam a padronização de suas definições e também comunicações. Presente também no desenvolvimento do contexto industrial, o autor defende um ecossistema capaz unir todos os dispositivos de forma intuitiva sem serem necessárias adaptações para todos os padrões proprietários. Esta ideia é o princípio para a elaboração
deste estudo, o qual possui seus objetivos discutidos a seguir.

\section{Objetivos Gerais e Secundários}
\label{sec:objetivos}

O objetivo geral deste trabalho é o desenvolvimento de um sistema capaz de adquirir dados provenientes de processos e interagir com eles, tratá-los e armazená-los em uma estrutura com alta estabilidade e confiabilidade, oferecendo todos os recursos necessários para a utilização do mesmo, sem ser necessárias quaisquer configurações avançadas em servidores ou outros serviços como em  \textit{softwares} disponíveis no mercado.
\newpage
Outros objetivos secundários podem ser alcançados através deste trabalho, como:

\begin{alineascomponto}
	\item a inclusão de protocolos nativos a dispositivos baseados em internet das coisas e também comuns aos utilizados em processos industriais que passam por esta transição;
    \item desenvolvimento de ferramentas colaborativas para que estudantes possam propor novas funcionalidades ou lógicas de uso mais eficientes e aproveitar as já existentes para a potencialização de trabalhos científicos.
\end{alineascomponto}

\section{Histórico de Desenvolvimento e Produção Científica}
\label{sec:historico-producao}

No ano de 2016, através de Iniciação Científica Voluntária, foi dado início ao desenvolvimento de um sistema de telemetria dinâmico capaz de adquirir informações de um sistema automotivo e disponibilizá-las através de visualização na internet. Concluído no ano de 2017, foi apresentado na Sessão de Painéis no XXIII Seminário de Iniciação Científica da Universidade Federal do Piauí e posteriormente obtida uma publicação relevante:

\begin{alineascomponto}
\item ROCHA NETO, W. B.; MENEZES JUNIOR, J. M. P.; SOUSA, R. V. L. C.
Análise de dados obtidos através de um sistema de telemetria automotivo utilizando K-NN.\textit{ Anais do XIV Encontro Nacional de Inteligência Artificial e
Computacional} (ENIAC’2017), Uberlândia-MG, 2017.
\end{alineascomponto}

No final do ano de 2018 foi retomado o desenvolvimento deste sistema, com o objetivo de adaptá-lo ao uso de outras áreas da Engenharia Elétrica, tendo sido implementado, por exemplo, no desenvolvimento de um controlador de irrigação parametrizado e controlado pela internet com objetivos acadêmicos.

Por fim, no início do ano de 2019, com o objetivo de expandir sua utilização, foi desenvolvido o sistema objeto deste trabalho para comportar múltiplos projetos na mesma plataforma, construídos através de simples interações com uma interface de gerenciamento.


\section{Organização do Trabalho}
\label{sec:organizacao-trabalho}

O restante deste trabalho está dividido conforme exposto a seguir. O Capítulo 2 é destinado às definições de dispositivos e protocolos que serão utilizados para desenvolver a ideia de como funcionam os processos, os dispositivos capazes de adquirir informações. Assim como os métodos e formatos de dados aos quais são transmitidas estas informações para uma interface de gerenciamento ou um servidor de dados.

O Capítulo 3 traz discussões sobre o conceito de Sistemas de Supervisão e Aquisição de Dados, apresenta vantagens e desvantagens destes sistemas, define suas variações existentes, oferece uma visão detalhada sobre como os dispositivos e protocolos do Capítulo 2 são utilizados para obtenção das informações e compara sistemas proprietários e de código aberto já disponíveis no mercado para este fim.

No Capítulo 4, são definidos o modelo de distribuição do \textit{software} desenvolvido, protocolos compatíveis e como o módulo de aquisição de dados utiliza-os, as vantagens e desvantagens do modelo implementado, lógica e hierarquia do sistema, como são armazenados os dados recebidos, além de informações sobre segurança e recursos computacionais esperados.

No Capítulo 5 é demonstrada a interface de gerenciamento desenvolvida, toda a metodologia de utilização do sistema e o detalhamento de todas as funcionalidades disponíveis.

No Capítulo 6 são desenvolvidos exemplos de utilização do sistema proposto, demonstrando a integração do mesmo com quatro plataformas distintas e discutindo os resultados obtidos. Uma comparação é feita entre um sistema similar disponível no mercado e o desenvolvido neste projeto.

Por fim, o Capítulo 7 é constituído por uma conclusão de tudo que foi discutido, apresenta a idealização do sistema como uma plataforma estudantil e os futuros trabalhos à serem desenvolvidos.
	\chapter{Dispositivos e Protocolos}
\label{chap:dispositivos-protocolos}

Nesta seção, são introduzidas tecnologias existentes para supervisão de processos e aquisição de dados e os protocolos mais utilizados para isto. Vantagens e desvantagens são comentadas para justificar a escolha do método deste trabalho.

    \section{Automação Industrial}        
    \label{sec:automacao-industrial}
    
    Para o aumento de produtividade de processos, a indústria utiliza diversas técnicas com o objetivo de introduzir máquinas eletromecânicas para a realização de tarefas que demandariam enorme esforço muscular e mental humanos. Além de oferecer um menor custo devido o aumento da capacidade de produção, essas técnicas acabam também por uniformizar o produto final e aumentar sua qualidade. Alguns conceitos utilizados na Automação Industrial e dispositivos empregados serão melhor descritos nesta seção.
    
    \subsection{Interface Humano-Máquina}
    \label{sec:ihm}

        A \gls{IHM} é uma ferramenta capaz de oferecer um aspecto visual de um ou mais processos à ela associados e, por meio de telas, fornece informações detalhadas sobre ele(s). Pode possuir teclado ou outras ferramentas para a interação do usuário com o processo final através de programas instalados no(s) dispositivo(s) \cite{mamede-instalacoes}. A Figura \ref{fig:figura-ihm} traz um exemplo de uma \gls{IHM} desenvolvida pela fabricante Branqs, que possui tela de 15 polegadas resistiva e colorida, entradas e saídas digitais integradas, além de outras funções que podem ser utilizadas para fornecer ao operador monitoramento e controle locais do processo em que esteja instalada \cite{Branqs}.
        
        \begin{figure}[!h]
		\Caption{\label{fig:figura-ihm} Exemplo de IHM da fabricante Branqs.}
		%\centering
		\UFCfig{}{
			\fbox{\includegraphics[width=8cm]{figuras/figura-ihm.jpg}}
		}{
			\Fonte{\cite{Branqs}}
		}	
	    \end{figure}
        
    \subsection{Unidade de Aquisição de Dados}
    \label{sec:uad}

        \gls{UAD} são dispositivos que recebem informações relativas ao processo em que estão inseridas e as transferem à um controlador de processo ou diretamente ao sistema de supervisão e controle, onde serão processadas e organizadas para exibição  \cite{mamede-instalacoes}. Dividem-se em duas categorias mais específicas:
        
        \begin{alineascomponto}
        	\item \gls{UD}
        	\item \gls{UADC}
        \end{alineascomponto}
        
    \subsubsection{Unidade Dedicada}
    \label{sec:ud}

    É um dispositivo inserido dentro do processo em que se mantenha apenas uma função dedicada  \cite{mamede-instalacoes}, como exemplos: relés digitais, intertravamento, etc.

    \subsubsection{Unidade de Aquisição de Dados e Controle}
    \label{sec:uadc}

        Tem a função de adquirir dados e controlar ações nos equipamentos respectivos, são compostos por cartões de eletrônicos associados cada um à uma função específica, unidades lógicas, memórias,  entradas e saídas de dados digitais ou analógicos \cite{mamede-instalacoes}. Dentre elas, as mais comuns são:
        
        \begin{alineascomponto}
        	\item Controlador Lógico Programável
        	\item Unidade Terminal Remota
        \end{alineascomponto}

    \subsubsubsection{Controlador Lógico Programável}
    \label{sec:clp}

       \gls{CLP} é a \gls{UADC} mais utilizada para controle de equipamentos através de programas desenvolvidos externamente pelo utilizador e nele gravados, simulando à nível de \textit{software} e substituindo: chaves, contatores, temporizadores, relés e outros dispositivos. Permitem a inclusão de cartões eletrônicos para a realização de diferentes tarefas específicas. Possui \gls{IHM}, onde o utilizador pode alterar a programação ou executar tarefas configuradas no \gls{CLP} \cite{mamede-instalacoes}. A Figura \ref{fig:figura-clp} mostra um \gls{CLP} da fabricante WEG, de modelo PLC300 que possui todas as características aqui descritas.
       
        \begin{figure}[!h]
		\Caption{\label{fig:figura-clp} Exemplo de CLP de modelo PLC300 da fabricante WEG.}
		%\centering
		\UFCfig{}{
			\fbox{\includegraphics[width=8cm]{figuras/clp-weg.jpg}}
		}{
			\Fonte{\cite{PLC300}.}
		}	
	    \end{figure}
	    
    \subsubsubsection{Unidade Terminal Remota}
    \label{sec:utr}

       \gls{UTR} é uma \gls{UADC} responsável por coletar informações e executar comandos de equipamentos do processo, sejam eles digitais ou analógicos. Possuem capacidade de executar programas em modo local independente do sistema de supervisão, ao mesmo tempo que possui capacidade de integração com o mesmo. Os comandos locais para equipamentos são feitos através de relés de maneira similar ao que ocorre no \gls{CLP}, por rotinas específicas armazenadas em programas gravados na própria \gls{UTR} \cite{mamede-instalacoes}. A Figura \ref{fig:figura-utr} mostra um \gls{UTR} da fabricante WEG, de modelo RUW-03 que possui todas as características aqui descritas.
       
        \begin{figure}[!h]
		\Caption{\label{fig:figura-utr} Exemplo de UTR de modelo RUW-03 da fabricante WEG.}
		%\centering
		\UFCfig{}{
			\fbox{\includegraphics[width=10cm]{figuras/utr-weg.jpg}}
		}{
			\Fonte{\cite{RUW03}}
		}	
	    \end{figure}
      

    \section{Protocolos de Comunicação}
    \label{sec:protocolos}
    A comunicação entre os dispositivos citados na seção anterior e outros, como: sensores, válvulas e atuadores em geral, é essencial para o funcionamento conjunto e ordenado dos mesmos. Desta forma, várias opções foram desenvolvidas ao longo do tempo para tornar a comunicação mais confiável, alguns dos protocolos mais utilizados atualmente são descritos nesta seção.
    
    \subsection{Modbus}
    \label{sec:modbus}
    
    Modbus é um protocolo de comunicação de dados voltado à automação industrial. Desenvolvido em 1979, pela \textit{Modicon}, é até hoje utilizado vastamente na indústria em \glspl{CLP} para comandos e aquisição de informações. Podem ser utilizados os padrões: RS-232, RS-485 ou Ethernet para a camada física de ligação, através de sinais discretos ou analógicos. É geralmente utilizado no tipo mestre-escravo, onde os escravos só enviam comunicação quando solicitadas pelo mestre  \cite{Modbus}.
    
    \subsubsection{Modbus Serial}
    \label{sec:modbus-serial}

        Em redes baseadas em RS-232 e RS-485, a comunicação do Modbus é feita de forma serial através de dois modos distintos: \gls{RTU} e \gls{ASCII}. A Figura \ref{fig:figura-modbus-serial} e Tabela \ref{tab:tabela-modbus-serial} representam os pinos da estrutura física RS-485 utilizado pelo Modbus Serial \cite{Modbus}.
        
        No \gls{RTU}, para cada byte transmitido, são codificados 2 caracteres. Os números variam entre -32768 e 32767, o tamanho da palavra RTU é de 8 bits.
        
        \begin{table}[h!]	
        	\centering
        	\Caption{\label{tab:tabela-modbus-rtu} Representação do pacote no modo RTU.}	
        	\IBGEtab{}{
        		\begin{tabular}{crrrr}
        			\toprule
        			Endereço Escravo & Código Função & Dados & CRC \\
        			\midrule \midrule
        			1 byte & 1 byte & 0 a 252 bytes & 2 bytes (CRC-16) \\
        			\bottomrule
        		\end{tabular}
        	}{
        	\Fonte{\cite{Modbus}.}
        }
        \end{table}
        
        No \gls{ASCII}, os dados são codificados com base na tabela \gls{ASCII}, em que cada byte é transmitido através de dois caracteres. O tamanho da palavra ASCII é de 7 bits, utilizando-se caracteres de intervalos 0-9 ou A-F e entre duas mensagens, 3-5 caracteres.
        
        \begin{table}[h!]	
        	\centering
        	\Caption{\label{tab:tabela-modbus-ascii} Representação do pacote no modo ASCII.}	
        	\IBGEtab{}{
        		\begin{tabular}{crrrrrr}
        			\toprule
        			Início & Endereço & Função & Dados & LRC & Final \\
        			\midrule \midrule
        			":" (ASCII 0x3Ah) & 2 caracteres & 2 caracteres & 0 a 2 x 252 caracteres & 2 caracteres & CR+LF (ASCII 0x0Dh + 0x0Ah) \\
        			\bottomrule
        		\end{tabular}
        	}{
        	\Fonte{\cite{Modbus}.}
        }
        \end{table}
        
        \begin{figure}[!h]
		\Caption{\label{fig:figura-modbus-serial} Pinagem do conector RS485 utilizado no protocolo Modbus Serial.}
		%\centering
		\UFCfig{}{
			\fbox{\includegraphics[width=10cm]{figuras/modbus.jpg}}
		}{
			\Fonte{se.com - Acessado em: 28/03/2019}
		}	
	    \end{figure}
	    
	   \begin{table}[h!]	
        	\centering
        	\Caption{\label{tab:tabela-modbus-serial} Pinagem do conector RS485 utilizado no protocolo Modbus Serial.}	
        	\IBGEtab{}{
        		\begin{tabular}{crrrrrr}
        			\toprule
        			Pino &&&&& Sinal \\
        			\midrule \midrule
        			1 &&&&& Não conectado \\
        			2 &&&&& RX - Recepção \\
        			3 &&&&& TX - Envio \\
        			4 &&&&& Não conectado \\
        			5 &&&&& Gnd - Terra \\
        			6 &&&&& Não conectado \\
        			7 &&&&& Não conectado \\
        			8 &&&&& Não conectado \\
        			9 &&&&& Não conectado \\
        			\bottomrule
        		\end{tabular}
        	}{
        	\Fonte{Adaptado de se.com - Acessado em: 28/03/2019}
        }
        \end{table}
        
    \subsubsection{Modbus TCP/IP}
    \label{sec:modbus-tcp}

        As redes baseadas em Ethernet, sob o protocolo \gls{TCP/IP}, se tornaram o método de transporte comum na Internet. O \gls{TCP/IP} é um conjunto de protocolos em camadas, que oferece confiabilidade no transporte de dados entre máquinas, e devido à isso, este padrão torna-se uma opção ideal para sistemas empresariais corporativos. O Modbus \gls{TCP/IP} tornou-se muito utilizado devido sua simplicidade e baixo custo, demandando \textit{hardwares} mínimos para ser utilizado. A maioria dos dispositivos Modbus atualmente presentes no mercado, suportam o padrão \gls{TCP/IP}, aumentando a cada ano a disponibilidade. Há também a possibilidade de conversão entre TCP/IP e Serial, onde é possível garantir a retrocompatibilidade entre dispositivos. A Figura \ref{fig:figura-modbus-ethernet} e Tabela \ref{tab:tabela-modbus-ethernet} trazem a representação do conector utilizado no Modbus \gls{TCP/IP} \cite{Modbus}.

        
        \begin{figure}[!h]
		\Caption{\label{fig:figura-modbus-ethernet} Pinagem do conector RJ45 utilizado no protocolo Modbus Ethernet.}
		%\centering
		\UFCfig{}{
			\fbox{\includegraphics[width=5cm]{figuras/modbus-ethernet.jpg}}
		}{
			\Fonte{Adaptado de se.com - Acessado em: 28/03/2019}
		}	
	    \end{figure}
	    
        \begin{table}[h!]	
        	\centering
        	\Caption{\label{tab:tabela-modbus-ethernet} Pinagem do conector RJ45 utilizado no protocolo Modbus Ethernet.}	
        	\IBGEtab{}{
        		\begin{tabular}{crrrrrr}
        			\toprule
        			Pino &&&&& Sinal \\
        			\midrule \midrule
        			1 &&&&& CAN\_H \\
        			2 &&&&& CAN\_L \\
        			3 &&&&& CAN\_GND \\
        			4 &&&&& D1 - RS485 (Modbus) \\
        			5 &&&&& D0 - RS485 (Modbus) \\
        			6 &&&&& Não conectado \\
        			7 &&&&& VP - Reservado ao conversor RS232/RS485 \\
        			8 &&&&& Comum \\
        			\bottomrule
        		\end{tabular}
        	}{
        	\Fonte{Adaptado de se.com - Acessado em: 28/03/2019}
        }
        \end{table}

        
    \subsection{OPC}
    \label{sec:opc}
    
        \gls{OPC}, inicialmente chamado \textit{Object Linking and Embedding for Process Control}, desenvolvido pela \textit{OPC Foundation} em 1996 e gerenciado por esta desde então, é o padrão de interoperabilidade para o transporte seguro e confiável de informações no espaço industrial, ele é independente de plataforma e garante o fluxo contínuo de informações entre dispositivos de vários fornecedores. É uma série de especificações desenvolvidas por fornecedores do setor, usuários e desenvolvedores. Essas especificações definem a interface entre Clientes e Servidores, bem como Servidores e Servidores, incluindo acesso a dados em tempo real, monitoramento de alarmes e eventos, acesso a dados históricos e outros aplicativos. \cite{OPC}
        
        Seu propósito inicial era agregar vários outros tipos de protocolos distintos de \glspl{CLP}, proprietários ou não, como: Modbus (seção \ref{sec:modbus}), Profibus, etc, de forma simplificada, para que \glspl{IHM} ou SCADAs (Capítulo \ref{chap:scada}) pudessem solicitar informações através de comunicação genérica, permitindo então, que os usuários implementassem sistemas usando os melhores produtos, interagindo perfeitamente via \gls{OPC}.

    \subsubsection{OPC Classic}
    \label{sec:opc-classic}

        Inicialmente, o padrão \gls{OPC} era restrito e baseado na plataforma \textit{Windows}, utilizando \gls{COM/DCOM} para a comunicação entre \textit{softwares}. Como visto na sigla original do \gls{OPC}, ele era suportado por \gls{OLE} voltado à Controle de Processo, essas especificações, agora conhecidas como \gls{OPC} \textit{Classic}, tiveram ampla adoção em vários setores \cite{OPCClassic}. Na Figura \ref{fig:figura-opc-classic} é representado funcionamento de um sistema que utilize \gls{OPC} \textit{Classic}, onde o Servidor \gls{OPC} recebe informações de inúmeros módulos de aquisição de dados, de diferentes fabricantes e diferentes protocolos e permite a interoperabilidade deles com o Cliente \gls{OPC}, de forma genérica.
                
        \begin{figure}[!h]
		\Caption{\label{fig:figura-opc-classic} Diagrama do funcionamento de um sistema que utilize OPC Classic.}
		%\centering
		\UFCfig{}{
			\fbox{\includegraphics[width=12cm]{figuras/opc.pdf}}
		}{
			\Fonte{O autor}
		}	
	    \end{figure}
	    
	    Existem três definições principais:
        
        \begin{alineascomponto}
        	\item Acesso a Dados - \textit{\gls{OPC} Data Access (DA)}: onde ocorrem troca de dados, incluindo valores, tempo e informações de qualidade.
        	\item Alarmes e Eventos - \textit{\gls{OPC} Alarms \& Events (AE)}: para troca de mensagens de alarmes e tipos de eventos, estados de variáveis e gerenciamento de estados.
        	\item Acesso a Dados Históricos - \textit{\gls{OPC} Historical Data Access (HDA)}: define os métodos de consulta e quais análises podem ser aplicadas a dados históricos, com registro de data e hora.
        \end{alineascomponto}

    \subsubsection{OPC-UA}
    \label{sec:opc-ua}

        Com a introdução de arquiteturas orientadas a serviços em sistemas de manufatura, surgiram novos desafios em segurança e modelagem de dados, a \textit{OPC Foundation} desenvolveu as especificações do \gls{OPC-UA} em 2008, sendo uma arquitetura orientada a serviços independentes, aberta e escalável.
        
	    \begin{figure}[!h]
		\Caption{\label{fig:figura-opc-ua} Diagrama do funcionamento de um sistema que utilize OPC-UA.}
		%\centering
		\UFCfig{}{
			\fbox{\includegraphics[width=10cm]{figuras/opc-ua.pdf}}
		}{
			\Fonte{O autor}
		}	
	    \end{figure}
        
        O \gls{OPC-UA} integra todas as funcionalidades do \gls{OPC} \textit{Classic}, além de outras melhorias, como:
        
        \begin{alineascomponto}
        	\item Segurança: criptografia de 128 ou 256 bits, verificação de erros para que a mensagem recebida seja exatamente a mensagem enviada, autenticação através de certificados e níveis de permissão.
        	\item Extensível: é possível adicionar novos recursos mantendo a compatibilidade com aplicações já existentes.
        	\item Descoberta: permite a busca por servidores \gls{OPC} na rede ou em computadores.
        	\item Hierarquia: todos os dados são dispostos de forma hierárquica, permitindo informações simples e complexas na mesma estrutura.
        	\item Auditoria: os dados à serem lidos/escritos possuem permissões de acesso tais como registros sobre sua utilização.
        	\item Independência de plataforma: funciona em computadores tradicionais e servidores em nuvem, seja o sistema operacional \textit{Linux}, \textit{Windows} ou outros, \glspl{CLP}, micro-controladores, etc.
        	
        \end{alineascomponto}
	    
    \subsection{HTTP}
    \label{sec:http}
    \gls{HTTP}, coordenado pela \gls{W3C}, é um protocolo de comunicação à nível de aplicação para distribuição de informação de hipermídia, é base para comunicação pela \gls{WEB} desde 1990. Inicialmente, em sua versão HTTP/0.9, era um simples protocolo de transferência de dados não tratados através da Internet e em sua versão atual HTTP/1.1, lançado em 1999, foram implementadas outras funcionalidades como a possibilidade de troca de mensagens no formato \gls{MIME}, que carregam consigo metainformações sobre a requisição ou resposta e o corpo das informações transferidas \cite{HTTP}.
    
    A transferência de informação acontece através de \textit{sockets} sob o protocolo \gls{TCP/IP}, onde com a arquitetura cliente-servidor, o cliente envia uma requisição ao servidor, com o padrão \gls{MIME} e  localizado através endereços, como o \gls{URI}, que identifica a informação acessada e \gls{URL}, que determina a localização desta informação, a conexão é completada e o servidor retorna o \textit{status} de acordo com o sucesso ou não da requisição e possíveis conteúdos também em formato \gls{MIME} caso sejam necessários, encerrando assim a conexão.
    
        \subsubsection{HTTPS}
        \label{sec:https}
        O \gls{HTTPS} é uma derivação do protocolo de comunicação \gls{HTTP} para mensagens seguras, projetado como uma camada de segurança utilizando o protocolo \gls{TLS}. Além de fornecer uma variedade de mecanismos de segurança para clientes e servidores, não são necessárias chaves públicas do lado cliente, suporta criptografia ponta a ponta e torna possível verificar a autenticidade do servidor através de certificados digitais. Seu uso é recomendado em redes inseguras, evitando a clonagem das informações trafegadas que poderia acontecer na ausência de criptografia \cite{HTTPS}. Segundo \cite{usoHTTPS}, em fevereiro de 2019, mais de 58.44\% dos 1 milhões \textit{websites} mais visitados da \textit{internet} já utilizavam o \gls{HTTPS}.
        
        \subsubsection{REST}
        \label{sec:rest}

        O \gls{REST} é um estilo de arquitetura para \textit{\gls{WEB} Service},  uma solução padronizada pela \gls{W3C} e \gls{OASIS} que busca fornecer interoperabilidade entre dispositivos e aplicações pela internet utilizando diferentes tipos de linguagens, o que a torna compatível com a maioria das aplicações já existentes \cite{W3C}.
        
        O envio e recebimento das mensagens é realizada de forma simplificada através dos protocolos \gls{HTTP} ou \gls{HTTPS} utilizando os formatos: \gls{XML}, \gls{JSON} ou \gls{HTML} e métodos de chamada bem definidos: GET, POST, PUT, PATCH e DELETE. Comumente utilizado por empresas no o desenvolvimento de \gls{API} para acesso a informações específicas sobre serviços, aplicações, faturas, etc. A Figura \ref{fig:figura-mqtt1} traz uma ideia geral sobre o funcionamento de uma API que utiliza a arquitetura REST para troca dos dados.
            
            \begin{figure}[!h]
    		\Caption{\label{fig:figura-rest1} Representação do funcionamento de uma API REST.}
    		%\centering
    		\UFCfig{}{
    			\fbox{\includegraphics[width=15cm]{figuras/figura-rest1.pdf}}
    		}{
    			\Fonte{O autor}
    		}	
    	    \end{figure}
    	
    	\subsubsection{Métodos de Chamada}
    	\label{sec:metodos-chamada}
    	
    	Quando uma nova requisição é feita, é necessário definir o método que será utilizado. Os métodos de chamada são padronizados para o protocolo \gls{HTTP} e são conhecidos como verbos, pois identificam a ação que será executada pela requisição. Os mais comuns são:
    	
        \begin{alineascomponto}
        	\item GET: Apenas recebe informações, as requisições devem ser seguras e idempotentes, ou seja, independente de quantas vezes ela seja repetida, com os mesmos parâmetros, o resultado sempre deve ser o mesmo. Podem haver solicitações parciais ou condicionais.

            \item POST: Envia e recebe informações, é utilizado na criação de novos "objetos" (elementos da aplicação), mas também é comum o uso para atualização destes.
            
            \item PUT: Envia e recebe informações, é utilizado na atualização de "objetos" já existentes, na falta do envio de algumas informações necessárias, estas são considerados nulas ou vazias. Assim como o GET, o PUT é idempotente.
            
            \item PATCH: Envia e recebe de informações, similar ao PUT,  é utilizado na atualização de "objetos" já existentes, porém, apenas os campos especificados.

            \item DELETE: Envia e recebe de informações, é utilizado na exclusão de "objetos", podendo ser imediato ou não.
        \end{alineascomponto}
        
        \begin{table}[h!]	
        	\centering
        	\Caption{\label{tab:tabela-metodos-chamada} Comparativo entre os métodos de chamada.}	
        	\IBGEtab{}{
        		\begin{tabular}{crrrr}
        			\toprule
        			Método & Descrição & Seguro & Idempotente \\
        			\midrule \midrule
        			GET & Recebe informações & sim & sim \\
        			POST & Cria objetos & não & não \\
        			PUT & Atualiza objetos, na falta de informações, considera como nulas & não & sim \\
        			PATCH & Atualiza objetos, alterando apenas as informações enviadas & não & não \\
        			DELETE & Exclui objetos, imediatamente ou não & não & sim \\
        			\bottomrule
        		\end{tabular}
        	}{
        	\Fonte{O autor}
        }
        \end{table}

    	\subsubsection{Formatos de Conteúdo}
    	\label{sec:formatos-de-conteudo}
    	São tipos de linguagem de marcação para necessidades especiais com a finalidade de transferência de informações pela \textit{internet}. Os mais comuns são:
    	
        \begin{alineascomponto}
        	\item \gls{HTML} (texto puro).
            \item \gls{XML}:  É baseado em texto simples, de simples leitura, pode representar listas, registros e árvores. Seu próprio formato descreve sua estrutura, campos e valores, os dados são organizados de forma hierárquica e é editável em qualquer ambiente \cite{XML}. A Figura \ref{fig:figura-xml} traz um exemplo de utilização do formato \gls{XML}.
                \begin{figure}[!h]
        		\Caption{\label{fig:figura-xml} Exemplo de informações organizadas no formato XML.}
        		%\centering
        		\UFCfig{}{
        			\fbox{\includegraphics[width=10cm]{figuras/exemploXML.png}}
        		}{
        			\Fonte{O autor}
        		}	
        	    \end{figure}
            \item \gls{JSON}: É um formato leve de informações, de simples leitura e análise. Assim como o \gls{XML} é hierárquico, em pares, ou seja, para cada rótulo, há um valor associado ou sub-conjunto destes. A Figura \ref{fig:figura-json} traz um exemplo de utilização do formato \gls{JSON} \cite{JSON}.
                \begin{figure}[!h]
        		\Caption{\label{fig:figura-json} Exemplo de informações organizadas no formato JSON.}
        		%\centering
        		\UFCfig{}{
        			\fbox{\includegraphics[width=11cm]{figuras/exemploJSON.png}}
        		}{
        			\Fonte{O autor}
        		}	
        	    \end{figure}
        \end{alineascomponto}

    \subsection{MQTT}
    \label{sec:mqtt}

        O \gls{MQTT} foi desenvolvido por Dr. Andy Stanford-Clark, da IBM, e Arlen Nipper, da Arcom no ano de 1999. É um protocolo de mensagens extremamente simples e leve, projetado para ser utilizado em dispositivos que tenham restrição de largura de banda, alta latência ou baixa confiabilidade. Baseia-se na topologia publicador/assinatura, onde as mensagens são enviadas com identificação através de tópicos (\textit{topics}) ou sub-tópicos, o que permite uma única mensagem ser destinada à múltiplos receptores com apenas um envio, ou da mesma forma, receber informações agrupadas de vários sub-tópicos. O elemento responsável pelo envio e recebimento de mensagens é denominado \textit{broker}, que funciona como uma central, intermediando as informações enviadas pelos dispositivos e aplicações da rede \cite{MQTT}. A Figura \ref{fig:figura-mqtt1} apresenta um exemplo de utilização ao qual um subscritor (possível dispositivo associado à rede) recebe informações de dois sensores utilizando um único tópico.
        
        \begin{figure}[!h]
		\Caption{\label{fig:figura-mqtt1} Representação do funcionamento do MQTT.}
		%\centering
		\UFCfig{}{
			\fbox{\includegraphics[width=15cm]{figuras/figura-mqtt1.pdf}}
		}{
			\Fonte{O autor}
		}	
	    \end{figure}
	    
	    A autenticação é feita através de usuário e senha, com a possibilidade de conexão criptografada e a escolha de três níveis de serviço (prioridades na transmissão) que dependerá do projeto em questão, qualidade de conexão do dispositivo, entre outros, sendo elas: 
	    
        \begin{alineascomponto}
        	\item Nível 0: Não é feita quaisquer confirmações sobre a entrega da informação, de forma que a mensagem é descartada após o envio.
        	\item Nível 1: São feitas várias tentativas de entrega até que se obtenha confirmação no recebimento, mesmo que isso implique no recebimento em duplicidade.
        	\item Nível 2: Há garantia de que a mensagem só será entregue uma vez, havendo tanto a confirmação de entrega da mensagem como a confirmação da confirmação de entrega.
        \end{alineascomponto}

\section{Síntese}
\label{sec:sintese-dispositivos-protocolos}

Neste capítulo são introduzidos os principais dispositivos e protocolos utilizados na indústria para atuação ou aquisição de dados referentes ao processo. Formatos de dados e outros conceitos aqui descritos, serão utilizados nos capítulos seguintes para entendimento da proposta deste trabalho.
	\chapter{Sistemas SCADA}
\label{sec:scada}

    Sistemas de Supervisão e Aquisição de Dados, do inglês, \gls{SCADA}, consistem basicamente de \textit{softwares} que monitoram e operam partes de um ou mais processos, através de unidades de aquisição de dados, como o \gls{CLP}, que por sua vez, conectado fisicamente ao servidor \gls{SCADA} e aos atuadores, utiliza protocolos de comunicação como os citados na seção \ref{sec:automacao-industrial} para obtenção e armazenamento destas informações. Com o domínio sobre as informações do processo, esta ferramenta é capaz de apresentar através de uma \gls{IHM} e de forma simplificada, valores e estados gerais do processo que se deseja atuar, desta forma, obtêm-se um maior controle sobre a tarefa, pois é possível centralizar a leitura de todos os sensores atuantes, a categorização e histórico destes dados, além da priorização de pendências do processo neste único sistema, reduzindo assim a necessidade um maior número de trabalhadores especializados que desempenhem a mesma função. \cite{WhatScada}
    
    O \gls{SCADA} apresenta uma série de vantagens, dentre elas: (i) redução de custos, devido a possibilidade de geração de relatórios detalhados úteis ao planejamento estratégico, evidenciando possíveis vícios do processo produtivo, (ii) maior desempenho na produção, por determinar os valores ótimos de trabalho, (iii) confiabilidade e continuidade, devido a existência de alarmes críticos, ou seja, notificações visuais ao operador, quando alguma variável ou condição do processo esteja em desacordo com o padrão de operação, desta forma, possíveis problemas que ocasionariam uma maior parada na produção, são mitigados com intervenções de forma quase imediata pelo operador caso sejam necessárias, trazendo assim vantagem competitiva.
    
    Todas as informações do processo podem coletadas e armazenadas em tempo real em um banco de dados, podendo serem implementadas no sistema de gestão empresarial da empresa ou utilizadas para cálculos mais complexos, sendo o último realizado por outras máquinas para garantir a que o \gls{SCADA} não tenha seu desempenho prejudicado.
    Uma representação básica do sistema \gls{SCADA} é ilustrada na Figura \ref{fig:figura-scada1}, ao qual todas as informações do processo são centralizadas e exibidas de forma simplificada ao colaborador. 
    
    \begin{figure}[!h]
	\Caption{\label{fig:figura-scada1} Arquitetura física básica do SCADA.}
	%\centering
	\UFCfig{}{
		\fbox{\includegraphics[width=14cm]{figuras/figura-scada1.pdf}}
	}{
		\Fonte{O autor}
	}	
    \end{figure}
    
\section{SCADA WEB}
\label{sec:scadaweb}

Rede Mundial de Computadores (\gls{WEB}), é como se designa o sistema de hiperligações e marcação de texto que permitem a disponibilidade de conteúdo através da \textit{internet}, como: páginas de texto, documentos, músicas, entre outros \cite{W3C}. O \gls{SCADA} \gls{WEB} é uma versão elaborada do SCADA convencional, onde os dados são transferidos para servidores na \textit{internet} e posteriormente processados, integrados às demais plataformas e/ou vistos em páginas WEB. A transição do SCADA para \gls{SCADA} \gls{WEB} ocorre principalmente devido à superação de uma baixa largura de banda e restrições de comunicação como ocorria antigamente. Os avanços tecnológicos possibilitaram a rápida expansão dos canais de dados através da \textit{internet}, onde até mesmo a transmissão de informações em tempo real, não é mais um fator limitante. \cite{ScadaWebSimp}

Sistemas \gls{SCADA} com base na \textit{internet} podem se tornar uma parte importante do funcionamento de sistemas de controle, onde o \gls{XML} e outras formatos disponíveis, podem oferecer possibilidade de resolução de problemas de incompatibilidade que existiam no \gls{SCADA} convencional, onde as fontes de comunicação com o processo são físicas e tornam necessários protocolos de comunicação específicos ou adaptações como o \gls{OPC}. Esta transição também ocorre com os dispositivos utilizados, como \glspl{CLP} e outros, que passam à contar com comunicação \gls{TCP/IP} integrada, alguns deles já com possibilidade de conexão \textit{Wi-Fi}, permitindo a utilização de protocolos como o \gls{HTTP} e \gsl{MQTT}, nativos da \gls{WEB}, diretamente do dispositivo, removendo grande parte da camada física. A Figura \ref{fig:figura-clp-wago} traz um exemplo da fabricante WAGO, com uma família de dispositivos que possuem conexão direta à internet, além de recursos de criptografia.

\begin{figure}[!h]
\Caption{\label{fig:figura-clp-wago} Família de Dispositivos com Comunicação Integrada da fabricante WAGO.}
%\centering
\UFCfig{}{
	\fbox{\includegraphics[width=10cm]{figuras/clp-wago.jpg}}
}{
	\Fonte{\cite{WAGO}}
}	
\end{figure}

Independente de como são obtidas as informações, a visualização se dá de forma mais familiar ao usuário, onde antes seria feita através de uma \gls{IHM} física, este ato se reduz à um simples \textit{smartphone} ou página no navegador simulando esta interface, sem ser necessária a instalação de qualquer \textit{software} adicional. A convergência das novas tecnologias, afeta drasticamente a supervisão destas informações, possibilitando controle distribuído e a possibilidade de armazenamento destas informações em qualquer lugar do mundo, com uma ampla capacidade de recursos \cite{ScadaWebInterOp}.

A implementação de um \gls{SCADA} \gls{WEB}, não só abre possibilidade de armazenamento de dados em várias localizações, como também eleva a capacidade de recursos computacionais à um nível muito superior, devido à possibilidade de utilização de servidores em nuvem - interligação de vários servidores através da internet formando um núcleo único de processamento - é possível controlar além de grupos de processos, grupos de plantas em único sistema. Segundo \cite{ScadaNextGer}, a nova geração do \gls{SCADA} pode modificar significamente a forma de projetar e implementar os processos industriais no futuro, onde o sistema terá que lidar com uma quantidade muito superior de dados distribuídos e informações em tempo real para tomar as decisões baseadas neste dados com informações internas e externas. A \gls{IHM} fica não mais limitada à um local físico, mas acessível através de todos os computadores, \textit{smartphones} e \textit{tablets} conectados à \textit{internet}, permitindo a colaboração simultânea na supervisão do processo. Com o uso de várias plantas simultâneamente, há a possibilidade de implementação de uma rotina por prioridades, dependendo das necessidades dinâmicas de cada cliente.  Poucas são as desvantagens de um sistema \gls{SCADA} \gls{WEB}, uma delas, seria a perda de robustez do sistema devido as informações e controle serem feitos à distância através da rede, tendo que serem considerados atraso em transporte que apesar de ser irrisório para a maioria das aplicações, podem chegar a ser um problema para outras. Uma representação básica do sistema \gls{SCADA} \gls{WEB} é ilustrada na Figura \ref{fig:figura-scada-web1}, onde as informações de um grupo de plantas são centralizadas e exibidas de forma simplificada ao colaborador.

    \begin{figure}[!h]
\Caption{\label{fig:figura-scada-web1} Arquitetura de um SCADA WEB.}
%\centering
\UFCfig{}{
	\fbox{\includegraphics[width=12cm]{figuras/figura-scada-web1.pdf}}
}{
	\Fonte{O autor}
}	
\end{figure}

\section{Sistemas SCADA disponíveis no mercado}
\label{sec:sistemas-scada}

\subsection{Sistemas proprietários}
\label{sec:sistemas-scada-proprietarios}

\subsubsection{Elipse E3}
\label{sec:elipse}

    O Elipse E3 \cite{Elipse}, desenvolvido pela empresa Elipse Software, representa a terceira geração do \gls{SCADA}. Utiliza o conceito de múltiplas camadas, onde incluem: servidores, regras de aplicação ou de negócio e estações clientes. O sistema é composto por 3 aplicações: 
    
    \begin{alineascomponto}
    	\item \textit{E3Server}: é o servidor das aplicações, em que se gerencia todos os processos de execução do \textit{software} e processa a comunicação entre eles. Suas ações são basicamente: envio das informações gráficas e dados para o cliente, gerenciamentos dos processos de E/S e comunicação com os diversos pontos de aquisição, controle da cópia de produtos, cliente e servidor OPC e sincronismo de alarmes e bases de dados. Permite também a distribuição deste serviço entre várias máquinas de acordo com a necessidade, com objetivo de manter a continuidade em uma eventual falha.
    	\item \textit{E3Viewer}: responsável pela interface de operação e visualização da aplicação que se encontra no E3Server, com operação local ou via \textit{intranet}/\textit{internet}, pode ser acessado por diversas plataformas como: Mac OS, Linux, Windows CE ou ainda, há a possibilidade de utilização do E3WebServer para gerenciamento adicional do acesso via Internet.
    	\item \textit{E3Studio}: ferramenta para configuração do sistema, servindo como plataforma universal do desenvolvimento. A configuração e execução compartilham da mesma base de dados, de forma que as edições das aplicações podem ser enviadas em \textit{runtime}, sem ser necessário a parada da aplicação, independente de ser feita local ou remotamente. É possível a edição de mais de um aplicativo ao mesmo tempo ou a edição ser feita por mais de uma pessoa devido compartilharem o mesmo servidor. Possui ferramentas, como: editores de telas, relatórios e \textit{scripts}.
    \end{alineascomponto}
    
    \begin{figure}[!h]
	\Caption{\label{fig:figura-elipse-e3} Demonstração da tela de processo do \textit{software} Elipse E3.}
	%\centering
	\UFCfig{}{
		\fbox{\includegraphics[width=15cm]{figuras/elipse-e3.png}}
	}{
		\Fonte{\cite{Elipse}.}
	}	
    \end{figure}

    
    Outras informações importantes:
    
    \begin{alineascomponto}
    	\item Possui drivers para comunicação com mais de 300 tipos de dispositivos e sistemas, sejam eles proprietários ou \gls{OPC}, além de produzir drivers sob encomenda.
    	\item Possui interfaces específicas para Access (.MDB), SQL Server/MSDE, Oracle ou acesso genérico através de padrões ADO e ODBC, faz acesso à base de dados corporativas fazendo o interfaceamento entre o processo e sistemas administrativos, de produção, manutenção e gestão.
    \end{alineascomponto}
    
\subsubsection{InduSoft Web Studio®}
\label{sec:indusoft}

    O InduSoft Web Studio® \cite{InduSoft}, desenvolvido pela empresa InduSoft, fornece componentes básicos de automação para o desenvolvimento de \glspl{IHM}, sistemas \glspl{SCADA} e soluções de instrumentação embarcada. O sistema é composto por 2 aplicações:
    
    \begin{alineascomponto}
        \item \textit{Server}: é o servidor das aplicações, em que se gerencia todos os processos de execução do \textit{software} e processa a comunicação entre eles. Suas ações são basicamente: envio das informações gráficas e dados para o cliente, gerenciamentos dos processos de entrada e saída e comunicação com os diversos pontos de aquisição, controle da cópia de produtos, servidor OPC e sincronismo de alarmes e bases de dados.
    	\item \textit{IoTViewer}: responsável pela interface de operação e visualização da aplicação que se encontra no \textit{Server}, com operação local ou via \textit{intranet}/\textit{internet}.
    \end{alineascomponto}

    \begin{figure}[!h]
	\Caption{\label{fig:figura-indusoft} Demonstração da tela de processo do \textit{software} InduSoft Web Studio®.}
	%\centering
	\UFCfig{}{
		\fbox{\includegraphics[width=15cm]{figuras/indusoft.png}}
	}{
		\Fonte{\cite{InduSoft}.}
	}	
    \end{figure}
    
    Outras informações importantes:
    
    \begin{alineascomponto}
    	\item A aplicação \textit{Server} suporta as plataformas \textit{Microsoft}, como: \textit{Windows CE, Mobile, XP Embedded e Server}, enquanto a aplicação cliente, o \textit{IoTView}, pode também suportar plataformas, como: \textit{Linux e VXWorks}.
    	\item Permite visualização de processo através de Navegador \gls{WEB}, podendo ser acessado através de celulares ou computadores de mesa, seja em rede local ou pela \textit{internet}.
    	\item Possui suporte para \gls{CLP} ou controlador e drivers para comunicação com mais de 200 tipos de dispositivos e sistemas, sejam eles proprietários ou \gls{OPC}, além de comunicação por \gls{TCP/IP}.
    	\item Alarmes podem ser enviados via \textit{e-mail}, celulares ou através do próprio navegador.
    	\item Permite acesso à base de dados corporativas fazendo o interfaceamento entre o processo e sistemas administrativos, de produção, manutenção e gestão.
    \end{alineascomponto}


\subsection{Sistemas de código aberto}
\label{sec:scadaweb}

\subsubsection{ScadaBR}
\label{sec:scadabr}

    O ScadaBR \cite{ScadaBR} é um \textit{software} livre e de código-fonte aberto. Abrange profissionais de automação, universidades, escolas técnicas e empresas de todos os portes. O projeto foi iniciado em 2006, por iniciativa da empresa MCA Sistemas com sede em Florianópolis - SC, que com o auxílio de outras empresas, a fundação CERTI e a Universidade Federal de Santa Catarina - UFSC, desenvolveram o sistema de forma completa em português baseado no \textit{software} Mango. Possui apoio da FINEP, SEBRAE e CNPq, que também, financiaram a iniciativa durante 2 anos.

    \begin{figure}[!h]
	\Caption{\label{fig:figura-indusoft} Demonstração de telas, incluindo a de processo, do \textit{software} ScadaBR.}
	%\centering
	\UFCfig{}{
		\fbox{\includegraphics[width=15cm]{figuras/scadabr.jpg}}
	}{
		\Fonte{\cite{ScadaBR}.}
	}	
    \end{figure}
    
    Outras informações importantes:
    
    \begin{alineascomponto}
	    \item A aplicação \textit{Server} suporta diferentes plataformas, como: \textit{Windows} 32/64 bits e \textit{Linux}.
	    \item Permite visualização de processo através de Navegador \gls{WEB}, podendo ser acessado através de celulares ou computadores de mesa, seja em rede local ou pela \textit{internet}.
    	\item Possui mais de 20 protocolos de comunicação, como: Modbus \gls{TCP/IP} e Serial, \gls{HTTP}, etc.
    	\item Customização de \textit{scripts} para controle, automação, etc.
    	\item Possibilidade de cálculos com funções matemáticas, estatísticas etc, com as variáveis do processo.
    	\item Níveis de permissão de usuários, com controle de acesso.
    \end{alineascomponto}
    
\subsubsection{TANGO Controls}
\label{sec:tango}

    O TANGO Controls \cite{Tango} é um \textit{software} livre e de código-fonte aberto. Foi desenvolvido pelo \textit{European Synchrotron Radiation Facility} em Genebra, França e seu desenvolvimento já supera 20 anos de duração. Foi desenvolvido principalmente para necessidades de instalações de pesquisas, com o conceito agregado de ser criado um novo \textit{framework}. Pode ser executado de forma autônoma ou distribuída, local ou remota.

    \begin{figure}[!h]
	\Caption{\label{fig:figura-tango} Demonstração da tela de processo do \textit{software} TANGO Controls.}
	%\centering
	\UFCfig{}{
		\fbox{\includegraphics[width=15cm]{figuras/tango.png}}
	}{
		\Fonte{\cite{Tango}.}
	}	
    \end{figure}
    
    Outras informações importantes:
    
    \begin{alineascomponto}
	    \item Permite visualização de processo através de Navegador \gls{WEB}, podendo ser acessado através de celulares ou computadores de mesa, seja em rede local ou pela \textit{internet}.
    	\item Possui diversos \textit{drivers} de comunicação, disponibilizados de forma gratuita.
    	\item Permite a adição de funções analíticas para a tomada de decisões.
    	\item Encontra-se em fase de transição para atender a demanda \textit{IoT} industrial.
    \end{alineascomponto}
    
\subsubsection{Rapid SCADA}
\label{sec:rapidscada}

    O Rapid SCADA \cite{RapidSCADA} é um \textit{software} livre e de código-fonte aberto, desenvolvido pela empresa russa \textit{Rapid Software}.

    \begin{figure}[!h]
	\Caption{\label{fig:figura-rapidscada} Demonstração da tela de processo do \textit{software} Rapid SCADA.}
	%\centering
	\UFCfig{}{
		\fbox{\includegraphics[width=15cm]{figuras/rapidscada.png}}
	}{
		\Fonte{\cite{RapidSCADA}.}
	}	
    \end{figure}
    
    Outras informações importantes:
    
    \begin{alineascomponto}
        \item É suportado por plataformas \textit{Windows} e \textit{Linux}.
	    \item O sistema possui uma interface de administração no modelo cliente/servidor e outra de monitoramento no formato \gls{WEB}.
	    \item Possui \textit{drivers} de comunicação, disponibilizados de forma gratuita, como: Modbus, \gls{OPC}, \gls{MQTT}, etc.
	    \item Níveis de permissão de usuários, com controle de acesso.
	    \item Alarmes de fogo e segurança, com avisos via interface.
	    \item A empresa cobra por serviços de treinamento e suporte na ferramenta, além de comercializar módulos de software adicionais.
    \end{alineascomponto}
    
\section{Bancos de Dados}
\label{sec:bancos-de-dados}

    \subsection{Relacionais}
    \label{sec:bancos-de-dados-relacionais}
        Oracle, MySQL, PostgreSQL
    
    \subsection{Não-Relacionais}
    \label{sec:bancos-de-dados-relacionais}
    \cite{cattell2011scalable}
    
        MongoDB, CouchDB e Cassandra
	\chapter{Sistema Proposto}
\label{chap:sistema-proposto}

Como exposto no Capítulo \ref{chap:dispositivos-protocolos}, são necessários vários protocolos ou adaptações para que a comunicação de um processo consiga atingir interoperabilidade entre o servidor e todos os dispositivos. Os sistemas \gls{SCADA} demonstrados no Capítulo \ref{chap:scada} utilizados para receber estas informações do servidor \gls{OPC} local ou diretamente destes dispositivos e organizá-las em um banco de dados, é a forma predominante na indústria e a forma mais confiável hoje. Todos os \textit{softwares} disponíveis para utilização que foram citados têm em comum seu modo de funcionamento, onde é instalado em uma máquina próxima e fisicamente ligada ao processo, que por sua vez mantém todos os serviços necessários para o funcionamento integrado dos módulos que o compõe. Alguns dispõem de integração com o protocolo \gls{MQTT} e interfaceamento \gls{WEB}, mas não são nativamente desenvolvidos para o funcionamento remoto.

Este trabalho propõe o desenvolvimento de um sistema \gls{SCADA} \gls{WEB} em que sua distribuição não seja mais como os sistemas \gls{SCADA} citados, na forma de um Produto como Serviço onde o cliente paga por um \textit{software} desenvolvido e futuras atualizações que venham ocorrer mas fornece toda a estrutura necessária para o funcionamento dele, e sim \textit{Software} como Serviço, do inglês, \gls{SaaS}, em que a própria plataforma fornece os recursos necessários para a disponibilização de todos os módulos do sistema, sejam eles: servidores, segurança e atualizações do próprio \textit{software}.

Com a premissa de que toda a estrutura esteja disponível através da internet, além de todas as funcionalidades de um sistema \gls{SCADA} convencional, várias vantagens podem ser listadas como:

\begin{alineascomponto}
    \item a capacitação profissional que antes seria necessária para a operação dessa estrutura é extremamente simplificada ao manuseio da interface;
    \item o gerenciamento é feito exclusivamente através do navegador podendo ser utilizado por todas as plataformas existentes na empresa, desde computadores, à \textit{smartphones} e \textit{tablets};
    \item com o uso da computação em nuvem, é possível a escalabilidade de recursos, sejam eles armazenamento ou processamento e tarefas simples que demandariam mais servidores ou a parada da aquisição de dados, podem agora ser feitas diretamente pela plataforma sem haver prejuízos ao processo;
    \item o mesmo sistema pode gerenciar múltiplos processos utilizando a mesma estrutura, podendo ser ou não apresentados na mesma interface;
    \item envio e recebimento de dados do processo podem ser feitos diretamente pelos dispositivos citados no Capítulo \ref{chap:dispositivos-protocolos} para a plataforma remota, desde que estejam disponíveis estas funcionalidades;
    \item com a possibilidade de uso de \gls{API} \gls{HTTP}, outros sistemas utilizados pela empresa podem trabalhar diretamente com o sistema \gls{SCADA}, obtendo informações ou atuando sobre o processo, se necessário e desejado.
\end{alineascomponto}

Algumas desvantagens também são conhecidas inicialmente devido seu modo de funcionamento, como:

\begin{alineascomponto}
    \item existência de um tempo de atraso na casa de milisegundos entre o envio e o recebimento das informações entre servidor e cliente devido a estrutura ser concebida de forma remota, relativamente longe do processo ao que seria a estrutura convencional;
    \item a dependência de uma boa conexão com a internet e estabilidade desta para a utilização do sistema, que pode ser amenizada caso os dispositivos utilizados tenham uma memória local capaz de armazenar os dados no caso de uma queda na conexão por uma janela de tempo suficiente até o retorno desta;
    \item dispositivos que não tenham nativamente acesso à internet deverão contar com drivers de comunicação capazes de intermediar estas informações ao sistema, como já acontece no \gls{SCADA} tradicional.

\end{alineascomponto}

\section{Organização e Hierarquia dos Projetos}
\label{sec:projetos}
No sistema \gls{SCADA} proposto, são definidos alguns conceitos:

\begin{alineascomponto}
    \item Variável: um conjunto de informações enviadas pelos dispositivos do processo ao módulo de aquisição de dados, que criam uma linha temporal no banco de dados do sistema.
    \item Objetos: caixas para conteúdo visual, que utilizarão as informações contidas nas Variáveis para apresentá-las por gráficos, tabelas, texto estático ou permitir a interação com o processo através de chaves, botões ou outros recursos.
    \item Projeto: estrutura lógica ao qual serão organizados Objetos de forma intuitiva na interface \gls{WEB} para compor todas as informações necessárias para apresentação do processo trabalhado.
    \item Desenvolvedor do Projeto: usuário principal do Projeto que possua permissão de inserção ou manutenção da organização de Objetos, gerenciamento de Variáveis e outras funções mais restritas.
    \item Operador: usuário com permissão apenas de visualizar as informações e/ou interagir com o processo através de chaves, botões e outros recursos.
\end{alineascomponto}

O sistema permite a criação ou visualização de múltiplos Projetos na mesma conta de forma totalmente isolada, baseado no exemplo de indústrias que dependam de mais de um processo ou tenham setores bem definidos que necessitem de interfaces de gerenciamento diferentes. Para que um novo Projeto seja constituído, o Desenvolvedor do Projeto o insere na plataforma através da interface de gerenciamento, cadastra variáveis relativas ao processo considerando seus tipos e em seguida cria novos Objetos necessários, cada Objeto poderá ter associado uma ou mais variáveis, para que o conteúdo destas ganhem forma, seja em forma de gráficos, tabelas ou para permitir o interação e controle do processo à um Operador. Estes objetos possuem configuração específica ao seu tipo, como o tamanho da janela de tempo que serão mostradas as informações por exemplo. Após toda a organização do Projeto, o Desenvolvedor do Projeto pode cadastrar os Operadores que irão monitorar e operar a interface de gerenciamento. Na Figura  \ref{fig:figura-projetos} é apresentado um esquemático de como seria a hierarquia desses elementos em relação ao Projeto. 

        \begin{figure}[!h]
		\Caption{\label{fig:figura-projetos} Lógica e hierarquia dos projetos desenvolvidos no sistema.}
		%\centering
		\UFCfig{}{
			\fbox{\includegraphics[width=15cm]{figuras/projetos.pdf}}
		}{
			\Fonte{O autor}
		}	
    	\end{figure}

    \section{Armazenamento dos Dados}
    \label{sec:armazenamento-dados}
    
    Os dados enviados para a plataforma, são tratados e inseridos em um banco de dados relacional, contendo chave e valor, onde é dada para o usuário uma ideia de variável de programação para a chave. Também, são descritos tipos de variáveis em que sua utilização serão considerados, como exemplo de uma chave liga/desliga que poderá utilizar uma variável binária. Mais detalhes sobre os tipos de variável e o banco de dados utilizado são mostrados a seguir.
    
        \subsection{Tipos de Variáveis}
        \label{sec:tipos-variaveis}
            
        \begin{alineascomponto}
            \item Binária: conhecida também por variável booleana, são permitidos valores 0 ou 1, \textit{false} ou \textit{true} e, dentro da plataforma, utilizada para chaves liga/desliga ou botões;
            \item Numérica: valores numéricos inteiros ou racionais, positivos ou negativos, que não ultrapassem o valor de 15 algarismos significativos ou 3 casas decimais;
            \item Texto: similar à variável \textit{string} de linguagens de programação, onde pode assumir qualquer valor com um tamanho máximo de 254 posições de texto.
        \end{alineascomponto}
        
        \subsection{Banco de Dados}
        \label{sec:banco-dados}
        O banco de dados utilizado para este projeto foi o \textit{software} MariaDB, desenvolvido pela MariaDB Foundation, é um dos projetos de bancos de dados mais populares do mundo, possui código aberto baseado no \textit{MySQL} e, por ser um banco relacional, seu uso é feito através de Linguagem de Consulta Estruturada, do inglês, \gls{SQL}, possuindo suporte para os tipos de dados mais comuns. É utilizado por grandes empresas, como: Wikipedia, Google, Booking.com, Alibaba.com e Microsoft \cite{MariaDB}. Na Figura \ref{fig:figura-mqtt-servico} é apresentado o logotipo do projeto utilizado.
        
        \begin{figure}[!h]
		\Caption{\label{fig:figura-mariadb-servico} Projeto de banco de dados de código aberto MariaDB.}
		%\centering
		\UFCfig{}{
			\fbox{\includegraphics[width=5cm]{figuras/mariadb.png}}
		}{
			\Fonte{\cite{MariaDB}}
		}	
    	\end{figure}
    	
\section{Aquisição de Dados}
\label{sec:aquisicao-dados}
O módulo mais importante para o funcionamento deste sistema é a aquisição de dados, pois, através dele se dará o direcionamento de todos os outros módulos. Conforme descritos no Capítulo \ref{chap:dispositivos-protocolos}, os protocolos \gls{HTTP} e \gls{MQTT} são  projetados para \gls{WEB} e possuem compatibilidade com diversos formatos de dados, sendo este o motivo para escolha deles neste projeto. Será feita uma abstração da camada física de dispositivos que utilizem \gls{OPC} por exemplo, partindo do pressuposto que estes possuam drivers de comunicação que possam fazer envio destes dados pela internet, ou seja, não haverá discriminação sobre os dados recebidos e tratados na plataforma. O utilizador da plataforma poderá escolher entre os dois protocolos de acordo com sua aplicação de interesse, abaixo são detalhados os processos referentes à aquisição de dados de cada um deles.

        \subsection{HTTP}
        \label{sec:aquisicao-http}
        O protocolo \gls{HTTP} é utilizado para a construção de uma \gls{API} baseada em \gls{REST} que possua a maior compatibilidade possível entre os métodos de chamada, para que até dispositivos simples possam trocar informações mesmo utilizando métodos de chamada triviais. Serão aceitos aqui, os três formatos de conteúdo descritos na seção \ref{sec:formatos-de-conteudo} devido o suporte nativo deste protocolo.
        
        Basicamente o processo de envio ou gerenciamento das informações contidas na plataforma, serão organizadas em 4 etapas: (i) Validação das Informações, (ii) Autenticação, (iii) Identificação das Informações, (iv) Banco de Dados, que serão detalhadas individualmente adiante. Na Figura \ref{fig:figura-http-geral} é apresentado um diagrama representando a sequência lógica destas quando iniciada a requisição.
        
        \begin{figure}[!h]
		\Caption{\label{fig:figura-http-geral} Diagrama das etapas do servidor para manipulação de dados.}
		%\centering
		\UFCfig{}{
			\fbox{\includegraphics[width=8cm]{figuras/http-geral.pdf}}
		}{
			\Fonte{O autor}
		}	
    	\end{figure}
    	
    	\begin{figure}[!h]
		\Caption{\label{fig:figura-http-validacao} Diagrama de validação das informações recebidas.}
		%\centering
		\UFCfig{}{
			\fbox{\includegraphics[width=15cm]{figuras/http-validacao.pdf}}
		}{
			\Fonte{O autor}
		}	
    	\end{figure}
    	
    	 Através de parâmetros configurados na \gls{URL} da \gls{API} da plataforma e cabeçalhos da requisição, o módulo identifica o método de chamada e qual comando o utilizador deseja. Se o comando enviado for reconhecido pelo módulo e exista alguma informação válida à ser enviada, é dado o direcionamento à etapa seguinte da Autenticação, caso contrário, a requisição é imediatamente encerrada. Na Figura \ref{fig:figura-http-validacao} é apresentado um diagrama com detalhes da etapa de validação das informações.
\newpage
    	 Posteriormente, dentre os parâmetros enviados é feita uma verificação do \textit{token}, um código que funciona como uma senha e, caso seja válido e esteja autorizado ao uso da \gls{API}, o módulo verifica seu limite de envios no caso de novas inserções de informações e segue à próxima etapa caso esteja apto, encerrando a requisição caso contrário. Na Figura \ref{fig:figura-http-autenticacao} é apresentado um diagrama detalhando a lógica da autenticação.
    	 
    	\begin{figure}[!h]
		\Caption{\label{fig:figura-http-autenticacao} Diagrama de autenticação do usuário.}
		%\centering
		\UFCfig{}{
			\fbox{\includegraphics[width=15cm]{figuras/http-autenticacao.pdf}}
		}{
			\Fonte{O autor}
		}	
    	\end{figure}
    	
    	 Na etapa de Identificação do tipo de informação enviada, basicamente são listadas todas as variáveis que se deseja inserir ou modificar no banco de dados e a categorização delas, sejam: númericas, binárias ou texto. Na Figura \ref{fig:figura-http-identificacao} é detalhada a identificação das informações provenientes da requisição.
    	
    	\begin{figure}[!h]
		\Caption{\label{fig:figura-http-identificacao} Diagrama sobre a identificação do tipo das informações.}
		%\centering
		\UFCfig{}{
			\fbox{\includegraphics[width=15cm]{figuras/http-identificacao.pdf}}
		}{
			\Fonte{O autor}
		}	
    	\end{figure}
    	
    	Por último, são verificadas uma a uma se o nome das variáveis já estão cadastradas no banco de dados e é dado o direcionamento para elas, repetindo a etapa de identificação para cada informação enviada, encerrando a requisição após o tratamento de todas elas. Na Figura \ref{fig:figura-http-banco} são apresentados detalhes sobre a etapa do banco de dados.
    	
    	\begin{figure}[!h]
		\Caption{\label{fig:figura-http-banco} Diagrama da etapa de banco de dados.}
		%\centering
		\UFCfig{}{
			\fbox{\includegraphics[width=15cm]{figuras/http-banco.pdf}}
		}{
			\Fonte{O autor}
		}	
    	\end{figure}
        
        \subsection{MQTT}
        \label{sec:aquisicao-mqtt}
        O protocolo \gls{MQTT} é utilizado para o envio e recebimento de informações de dispositivos que tenham restrição de largura de banda ou que tenham suporte nativo à sua utilização, devido sua facilidade de uso. Se faz necessária uma aplicação \textit{broker} para o funcionamento deste serviço e, devido ser um projeto de código aberto, foi escolhido o \textit{software} VerneMQ para esta finalidade.
        
        O projeto VerneMQ foi desenvolvido para ser distribuído, ou seja, há a possibilidade de escalabilidade vertical (quando se inclui mais servidores para a manutenção do serviço) e horizontal (quando se aumenta os recursos de um único servidor). Além disto, traz funcionalidades como: autenticação, criptografia, níveis de serviço, envio de mensagens \textit{offline}, balanceamento de recursos, entre outros \cite{VerneMQ}. Na Figura \ref{fig:figura-mqtt-servico} é apresentado o logotipo do serviço utilizado e as principais plataformas suportadas.
        
        \begin{figure}[!h]
		\Caption{\label{fig:figura-mqtt-servico} Projeto utilizado para manter o serviço do MQTT.}
		%\centering
		\UFCfig{}{
			\fbox{\includegraphics[width=10cm]{figuras/vernemq.png}}
		}{
			\Fonte{Adaptado de \cite{VerneMQ}}
		}	
    	\end{figure}
    	
    	Conforme explicado na seção \ref{sec:mqtt}, o MQTT recebe e envia informações através de tópicos e, no caso desta plataforma, para o envio de informações, são considerados: tópico como o \textit{token} do usuário e sub-tópico como a variável cadastrada no projeto em que se deseja inserir no banco de dados, simplificando portanto a autenticação deste serviço e tornando similar ao uso da \gls{API} \gls{HTTP}. O módulo de aquisição de dados para o MQTT funciona como um subscritor neste serviço, capturando as informações enviadas pelo usuário no \textit{broker} e inserindo-as no banco de dados caso sejam válidas, o retorno de informação é dado pelo mesmo canal de envio. Os níveis de qualidade de serviço para a entrega das mensagens podem ser utilizados à critério de projeto pelo usuário.

\section{Segurança}
    \label{sec:seguranca}
    Além do \textit{token} citado na seção \ref{sec:aquisicao-dados}, outros elementos e critérios de segurança são considerados para garantir a segurança dos dados manipulados dentro da plataforma, todos os critérios de segurança desenvolvidos neste projeto serão detalhados nas seções abaixo.
        
        \subsection{Criptografia}
        \label{sec:criptografia}
        Por se tratar de dados críticos e de potencial interesse comercial do cliente, optou-se pela implementação de criptografia ponta a ponta não somente na área de acesso da interface de gerenciamento, mas também no módulo de aquisição de dados. ficando a critério do cliente a sua utilização ou não, já que podem ser incrementados um pequeno atraso de tempo e recursos na sua utilização, perceptíveis em processos que necessitem de curtos intervalos de tempo para funcionamento.  Em ambos os módulos, gerenciamento e aquisição de dados, é utilizado o protocolo \gls{HTTPS} visto na seção \ref{sec:https}, que utiliza certificados fornecidos por autoridades de certificação, como exemplo a empresa \textit{VeriSign} entre outras que são autorizados previamente nos navegadores, tendo a certeza de uma conexão segura e autêntica é implementada a criptografia \gls{TLS} entre servidor e cliente e a requisição é concluída similarmente ao que seria no protocolo \gls{HTTP} comum, o mesmo acontece para o protocolo \gls{MQTT} quando necessária sua utilização.
        
        \subsection{Proteções}
        \label{sec:protecoes}
        
        Alguns métodos são implementados para garantir que não haja vazamento ou quaisquer problemas em relação aos dados armazenados, impede-se tentativas de acesso não autorizado ao sistema e/ou ações maliciosas, os métodos considerados para este projeto foram:
        
        \begin{alineascomponto}
            \item proteção contra força bruta, a qual ocorrem tentativas de acesso aleatórias até que se consiga um acerto dos dados de acesso corretos e para este caso, são implementadas medidas para limitar as tentativas errôneas de um único utilizador;
            \item controle de acesso por Endereço IP, para cada acesso válido realizado é identificado o Endereço IP do utilizador e associado à sua conexão, caso haja modificação deste, a conexão é encerrada imediatamente;
            \item injeções \gls{SQL}, no envio de informações para o módulo de aquisição de dados ou até mesmo na interface de gerenciamento, ocorre a tentativa de manipulação das consultas ao banco de dados com o objetivo de realizar modificações severas ou a tomada de controle do mesmo, para este caso são feitos tratamentos específicos para todos os dados de entrada feitos à plataforma;
            \item outras proteções passivas também são implementadas para evitar o abuso de acessos com o objetivo de indisponibilizar o serviço.
        \end{alineascomponto}

        \subsection{Controle de Acesso}
        \label{sec:controle-acesso}
        Os acessos à interface são feitos através de usuário e senha armazenados em banco de dados e criptografados previamente, de forma que toda ação feita dentro da plataforma é provida de auditoria, faz-se um registro com informações sobre data e hora, local de onde foi feito o acesso, entre outros. Neste caso, também são implementados níveis de acesso à interface de gerenciamento, podendo serem diferenciados utilizadores para um mesmo projeto dentro da plataforma. O módulo de aquisição de dados trabalha exclusivamente com autenticação por \textit{token}, gerado no momento em que é feita inclusão de um cliente em um projeto, mais detalhes sobre como este \textit{token} é gerado e fornecido pela interface serão dados adiante.
        
\section{Recursos Computacionais}
\label{sec:recursos-computacionais }
    Toda a aplicação baseia-se na utilização de computação em nuvem, em que servidores, armazenamento, redes e \textit{software} rodam sob uma camada lógica virtual em uma infraestrutura de servidores integrada por fornecedores que permite o rateio dos recursos entre utilizadores, possibilitando a redução do custo operacional do serviço, aumentando sua eficiência e confiabilidade já que uma falha física não afetará toda a estrutura.
    
    Conforme citado, o sistema foi desenvolvido para uma forma de distribuição \gls{SaaS}, o que significa que o usuário final não se preocupará com quaisquer aspectos de infraestrutura, ficando à cargo da equipe de operação do sistema \gls{SCADA} proposto a manutenção geral destes serviços. Para que não haja um desbalanceamento no uso e possível sobrecarga decorrente de abusos de um só usuário que afete os demais, o sistema foi desenvolvido para que cada conta haja um limite de envio de informações por hora e também a quantidade de dias que ocorra retenção desta informação em banco de dados, que podem variar de usuário pra usuário à critério de suas necessidades de projeto e da equipe do sistema \gls{SCADA}. Desta forma, é possível prever a utilização de recursos utilizados nos módulos e garantir a disponibilidade do sistema.
    
\section{Síntese}
\label{sec:sintese-sistema}

Neste Capítulo, foram introduzidos todos os aspectos gerais necessários para o desenvolvimento do sistema proposto. Tendo sido descritas todas as funcionalidades necessárias, o sistema foi devidamente desenvolvido e o capítulo seguinte traz o capturas de tela e instruções de uso em um passo-a-passo sobre o sistema real em funcionamento.
	\chapter{Interface de Gerenciamento: rscada}
\label{chap:interface-web}

A interface de gerenciamento foi desenvolvida conforme todos os conceitos descritos no Capítulo \ref{chap:sistema-proposto}, o tema utilizado para estilo da página foi desenvolvido pela empresa Colorlib com código-fonte aberto e modificado à critérios do projeto \cite{Concept}. O nome do sistema, rscada,  foi constituído da inicial R do nome do autor e o tipo do sistema em questão. Todo o código dos módulos foi programado utilizando a linguagem de programação PHP, com exceção da integração entre o banco de dados e o serviço VerneMQ, responsável pelo protocolo MQTT, teve por base programação LUA e os exemplos de utilização da plataforma disponíveis no Capítulo \ref{chap:resultados} que foram desenvolvidos utilizando C++ com pequenas modificações. Todas as etapas e telas do projeto são explicadas abaixo.

\section{Acesso ao Sistema}
\label{sec:acesso-sistema}
Ao acessar a interface de gerenciamento, são solicitados ao utilizador os dados de acesso, sendo eles: usuário ou e-mail e senha. A Figura \ref{fig:figura-rscada-1} traz a tela real de acesso do sistema desenvolvido.

        \begin{figure}[!h]
		\Caption{\label{fig:figura-rscada-1} Tela de autenticação da Interface Web.}
		%\centering
		\UFCfig{}{
			\fbox{\includegraphics[width=8cm]{figuras/rscada-1.png}}
		}{
			\Fonte{O autor}
		}	
    	\end{figure}

Caso seja um novo utilizador, é fornecido um botão na tela de acesso para criação de novas contas, onde o sistema redirecionará à página de criação de contas e solicitará ao novo usuário: Nome, E-mail, Usuário e Senha, além de uma confirmação de leitura sobre os termos de serviço do rscada. Ao submeter o formulário de criação de nova conta, é enviado ao email do usuário, um \textit{link} contendo um código de confirmação da conta para validar o e-mail digitado. A Figura \ref{fig:figura-rscada-2} demonstra a tela de cadastro para novas contas do rscada.

        \begin{figure}[!h]
		\Caption{\label{fig:figura-rscada-2} Tela de cadastro para novos usuários.}
		%\centering
		\UFCfig{}{
			\fbox{\includegraphics[width=8cm]{figuras/rscada-2.png}}
		}{
			\Fonte{O autor}
		}	
    	\end{figure}

\section{Tela inicial da interface}
\label{sec:tela-inicial}
Devidamente autenticado no sistema, o usuário é redirecionado à tela inicial onde o foco são os projetos cadastrados pertencentes à ele. Informações sobre a quantidade de Clientes associados aos projetos e botões de ações, são disponíveis na página, dentre elas: Novo Projeto, Gerenciamento dos Projetos existentes e Exclusão dos mesmos. Um menu é disponibilizado na lateral esquerda da tela, com os principais atalhos para funções do sistema, como: (i) Minha Conta, onde o usuário poderá alterar detalhes como: E-mail ou Senha, (ii) Meus Projetos, quando necessário retornar à tela inicial dos projetos, (iii) Meus Clientes, para o cadastro de clientes descritos anteriormente como Operadores, que farão uso dos projetos quando desenvolvidos, (iv) Meus Domínios, que o usuário poderá cadastrar o endereço de site o qual os clientes terão acesso ao projeto desenvolvido, (v) Alarmes, para visualizar eventos e alertas de informações que tenham sido inseridas no sistema e que não correspondem aos valores ideiais de funcionamento, entre outros. As Figura \ref{fig:figura-rscada-2} e \ref{fig:figura-rscada-5} trazem capturas da tela descrita.

        \begin{figure}[!h]
		\Caption{\label{fig:figura-rscada-3} Tela inicial do sistema após autenticação.}
		%\centering
		\UFCfig{}{
			\fbox{\includegraphics[width=15cm]{figuras/rscada-3.png}}
		}{
			\Fonte{O autor}
		}	
    	\end{figure}
    	
    	\begin{figure}[!h]
		\Caption{\label{fig:figura-rscada-5} Apresentação Geral de todos os projetos cadastrados.}
		%\centering
		\UFCfig{}{
			\fbox{\includegraphics[width=15cm]{figuras/rscada-5.png}}
		}{
			\Fonte{O autor}
		}	
    	\end{figure}

A plataforma também permite o ajuste automático ao tipo de tela que o utilizador tenha, é a função que abre possibilidade para que o sistema seja adaptado à qualquer tipo de plataforma. Partindo do princípio que todo dispositivo conectado à internet tenha um navegador ou forma primitiva de um, é possível sua utilização, como exemplo, a Figura \ref{fig:figura-rscada-smartphone} traz uma captura de tela quando aberta em um \textit{smartphone}.

    	\begin{figure}[!h]
		\Caption{\label{fig:figura-rscada-smartphone} Tela inicial do sistema após autenticação em um \textit{smartphone}.}
		%\centering
		\UFCfig{}{
			\fbox{\includegraphics[width=6cm]{figuras/rscada-smartphone.png}}
		}{
			\Fonte{O autor}
		}	
    	\end{figure}
    	
\section{Criação de novos projetos}
\label{sec:criacao-projetos}
Ao clicar no botão Novo Projeto, o usuário é redirecionado à uma página solicitando o nome que se deseja para ele, conforme a Figura \ref{fig:figura-rscada-4}. Ao submeter o formulário é apresentada uma mensagem de sucesso, caso seja possível a criação dele e, em seguida, encaminhado à página inicial do projeto, representado na Figura \ref{fig:figura-rscada-novo}. 

        \begin{figure}[!h]
		\Caption{\label{fig:figura-rscada-4} Página de cadastro de novos Projetos.}
		%\centering
		\UFCfig{}{
			\fbox{\includegraphics[width=15cm]{figuras/rscada-4.png}}
		}{
			\Fonte{O autor}
		}	
    	\end{figure}
    	
    	\begin{figure}[!h]
		\Caption{\label{fig:figura-rscada-novo} Página de gerenciamento de um novo projeto.}
		%\centering
		\UFCfig{}{
			\fbox{\includegraphics[width=15cm]{figuras/rscada-novo.png}}
		}{
			\Fonte{O autor}
		}	
    	\end{figure}

É apresentada uma mensagem de que não existe nenhuma variável cadastrada e há um reforço de cor no botão de criação para indicar a relevância dessa ação. Ao clicar no botão de Nova Variável, o usuário é redirecionado à uma tela ao qual poderá escolher o tipo pretendido, entre as citadas na seção \ref{sec:tipos-variaveis}, a variável iniciada por letra e minúscula, um nome que represente uma descrição à ela e a unidade correspondente caso exista. A Figura \ref{fig:figura-rscada-6} traz uma captura da tela de novas variáveis.

        \begin{figure}[!h]
		\Caption{\label{fig:figura-rscada-6} Página de cadastro de novas Variáveis.}
		%\centering
		\UFCfig{}{
			\fbox{\includegraphics[width=15cm]{figuras/rscada-6.png}}
		}{
			\Fonte{O autor}
		}	
    	\end{figure}
    	
Quando submetido o formulário, é apresentada uma tela de sucesso caso a variável seja inserida no banco de dados e então, o usuário poderá gerenciar todas as variáveis já cadastradas no projeto com informação adicional sobre a quantidade de informações já inseridas no banco de dados para aquela variável específica, conforme representado na Figura \ref{fig:figura-rscada-variaveis} e coluna Registros. O cadastro de novas variáveis também pode ser feito diretamente no módulo de aquisição de dados, onde submetendo informações citando uma variável não cadastrada, o próprio módulo identificará o tipo dela e fará a submissão ao banco de dados. Este procedimento é feito para evitar a perda de informação, caso o desenvolvedor considere novas variáveis no dispositivo do processo e não tenha atualizado no sistema \gls{SCADA} ainda.

        \begin{figure}[!h]
		\Caption{\label{fig:figura-rscada-variaveis} Gerenciamento das variáveis cadastradas no projeto.}
		%\centering
		\UFCfig{}{
			\fbox{\includegraphics[width=15cm]{figuras/rscada-variaveis.png}}
		}{
			\Fonte{O autor}
		}	
    	\end{figure}

Com as variáveis já cadastradas no sistema, é oferecida a opção de criação de novos Objetos, detalhados na seção \ref{sec:projetos}, que quando clicado o botão, o usuário é direcionado à página ilustrada pela Figura \ref{fig:figura-rscada-7}. São solicitados o tipo do objeto, entre as opções já disponíveis na data de apresentação deste trabalho os seguintes:

\begin{alineascomponto}
    \item Chave Binária: disponibiliza um botão que controla uma variável binária, à qual cada clique inverte seu valor, foi desenvolvida pensando em ser utilizada na função liga/desliga de alguma ação dentro do processo que seja necessário.
    \item Botão de Ação: disponibiliza um botão que pode oferecer o envio de um valor determinado em uma variável no instante em que for solicitado, foi desenvolvido pensando em ser utilizado em funções que não sejam sustentadas, ou seja, tenha um único acionamento com período de duração determinado.
    \item Último Valor: disponibiliza um texto fixo com o último valor enviado pelo dispositivo da variável desejada no objeto, é dada a variação em relação ao valor imediatamente anterior à ele e o tempo que se passou desde o último envio.
    \item Gráfico de Linha: disponibiliza na tela um gráfico de linhas com uma série temporal das informações enviadas à variável escolhida. 
    \item Gráfico de Área: disponibiliza na tela um gráfico de área com uma série temporal das informações enviadas à variável escolhida.
    \item Gráfico de Barras: disponibiliza na tela um gráfico de barras com uma série temporal das informações enviadas à variável escolhida.
    \item Tabela de Informações: disponibiliza na tela uma tabela contendo uma quantidade dos últimos valores das informações enviadas à variável escolhida.
    \item Tabela de Eventos:  disponibiliza na tela uma tabela contendo os registros de alarmes e alertas sobre envio de informações que não estejam em acordo com o definido para a variável escolhida, representando também os alertas visuais que serão oferecidos ao operador quando ocorridas. A Figura \ref{fig:figura-rscada-alarmes} traz exemplos dos alarmes que são emitidos em tela para o Operador.
\end{alineascomponto}

Em seguida, são solicitados: Título, Descrição e Tamanho do Objeto, que servirão para apresentação da caixa gráfica quando inserida na interface do projeto. O Tamanho é uma seleção entre: Pequeno, Médio, Grande e Gigante, sendo respectivamente relacionado à largura ocupada da tela pelo Objeto, 25\%, 50\%, 75\% e 100\%.

        \begin{figure}[!h]
		\Caption{\label{fig:figura-rscada-7} Página de cadastro de novos Objetos.}
		%\centering
		\UFCfig{}{
			\fbox{\includegraphics[width=15cm]{figuras/rscada-7.png}}
		}{
			\Fonte{O autor}
		}	
    	\end{figure}
    	
    	\begin{figure}[!h]
		\Caption{\label{fig:figura-rscada-alarmes} Alertas do sistema quando há uma não-conformidade da informação recebida.}
		%\centering
		\UFCfig{}{
			\fbox{\includegraphics[width=15cm]{figuras/rscada-alarmes.png}}
		}{
			\Fonte{O autor}
		}	
    	\end{figure}
    	
Quando criado o objeto, é solicitada a edição dele para a determinação dos parâmetros referentes ao tipo específico escolhido, conforme a Figura \ref{fig:figura-rscada-10}.
        
        \begin{figure}[!h]
		\Caption{\label{fig:figura-rscada-10} Objeto recém-criado.}
		%\centering
		\UFCfig{}{
			\fbox{\includegraphics[width=10cm]{figuras/rscada-10.png}}
		}{
			\Fonte{O autor}
		}	
    	\end{figure}
    	
Na tela referente à edição do Objeto, é selecionada a variável que o objeto manipulará e a Janela de Tempo que seria a quantidade de pontos de informação à serem consideradas pelo sistema quando gerar o Objeto na interface de gerenciamento. A Figura \ref{fig:figura-rscada-editar-objeto} traz um exemplo de edição de um objeto do tipo Tabela de Informações em uma Nova Variável e Janela de Tempo de 5 minutos.

    	
    	\begin{figure}[!h]
		\Caption{\label{fig:figura-rscada-editar-objeto} Edição de objeto para determinação de parâmetros.}
		%\centering
		\UFCfig{}{
			\fbox{\includegraphics[width=15cm]{figuras/rscada-editar-objeto.png}}
		}{
			\Fonte{O autor}
		}	
    	\end{figure}

Após a organização dos Objetos e Variáveis das etapas acima, deve ser feita a inclusão no menu Meus Clientes, dos operadores que utilizarão o projeto criado. A página Meus Clientes, disponibiliza um botão Novo Cliente, que após clicado, solicita dados do cliente, como: Nome, E-mail, Usuário e Senha que servirão para acesso do mesmo à interface de gerenciamento quando incluso no projeto. A Figura \ref{fig:figura-rscada-novo-cliente} traz a página de inclusão de novos clientes.

        \begin{figure}[!h]
		\Caption{\label{fig:figura-rscada-novo-cliente} Inserção de novo cliente ao sistema.}
		%\centering
		\UFCfig{}{
			\fbox{\includegraphics[width=15cm]{figuras/rscada-novo-cliente.png}}
		}{
			\Fonte{O autor}
		}	
    	\end{figure}
    	
A submissão do formulário com os dados do cliente, desde que corretamente inseridos no banco de dados, resultará numa página com uma mensagem de sucesso e o usuário será direcionado à tela de gerenciamento de todos os clientes cadastrados no sistema, outras ações são disponíveis também nesta seção como o gerenciamento do cliente, alteração de seus dados de cadastro e a exclusão do mesmo, conforme a Figura \ref{fig:figura-rscada-clientes}.
    	
    	\begin{figure}[!h]
		\Caption{\label{fig:figura-rscada-clientes} Gerenciamento dos clientes cadastrados no sistema.}
		%\centering
		\UFCfig{}{
			\fbox{\includegraphics[width=15cm]{figuras/rscada-clientes.png}}
		}{
			\Fonte{O autor}
		}	
    	\end{figure}
    	
Após inserido um novo cliente, é possível fazer a associação do mesmo ao projeto criado anteriormente. A Figura \ref{fig:figura-rscada-8} traz detalhes de como é realizada esta ação no sistema.
    	
        \begin{figure}[!h]
		\Caption{\label{fig:figura-rscada-8} Associação de novo cliente ao projeto.}
		%\centering
		\UFCfig{}{
			\fbox{\includegraphics[width=15cm]{figuras/rscada-8.png}}
		}{
			\Fonte{O autor}
		}	
    	\end{figure}

Desde que seja corretamente associado o cliente ao projeto e inserida esta informação no banco de dados, é retornada uma página de sucesso e o usuário é redirecionado à página de Clientes Associados, onde podem ser verificados o \textit{Token} de acesso para envio de informações, nível do cliente e quantidade de informações enviadas por ele. Estão disponíveis também outras ferramentas, como: Editar a associação entre cliente e projeto e, a exclusão do mesmo, conforme a Figura \ref{fig:figura-rscada-9} que traz detalhes da tela.

        \begin{figure}[!h]
		\Caption{\label{fig:figura-rscada-9} Visão geral dos clientes cadastrados no projeto.}
		%\centering
		\UFCfig{}{
			\fbox{\includegraphics[width=15cm]{figuras/rscada-9.png}}
		}{
			\Fonte{O autor}
		}	
    	\end{figure}
    	
\section{Síntese}
\label{sec:sintese-rscada}

A abordagem utilizada neste Capítulo, pode ser sintetizada por um Diagrama de Caso de Uso, descrevendo a sequência e as unidades de interação com o sistema, disponível na Figura \ref{fig:figura-diagrama-uso}. Tendo sido detalhadas todas as funcionalidades propostas desenvolvidas, o capítulo seguinte traz uma aproximação do uso real deste sistema, com base em exemplos de diversas situações de utilização do mesmo. Os dois protocolos de aquisição de dados, \gls{MQTT} e \gls{HTTP}, são utilizados para demonstrar a facilidade de uso e a capacidade de integração do sistema.

        \begin{figure}[!h]
		\Caption{\label{fig:figura-diagrama-uso} Diagrama de caso de uso sintetizando os níveis de permissão do sistema.}
		%\centering
		\UFCfig{}{
			\fbox{\includegraphics[width=15cm]{figuras/figura-diagrama-uso.pdf}}
		}{
			\Fonte{O autor}
		}	
    	\end{figure}
    	
	\chapter{Resultados}
\label{chap:resultados}

Texto texto texto

Texto texto  Referenciando a  texto texto texto 
	\chapter{Conclusões e Trabalhos Futuros}
\label{chap:conclusoes-e-trabalhos-futuros}

No Capítulo \ref{chap:sistema-proposto}, foi proposto o desenvolvimento de um sistema \gls{SCADA}, seguindo o modelo de distribuição \gls{SaaS}, fornecendo todos os elementos necessários para seu funcionamento. As vantagens e desvantagens do sistema proposto foram discutidas, assim como possíveis soluções para amenização destes problemas. A estrutura lógica base para o desenvolvimento foi detalhada, como: a hierarquia, tipos de variáveis, formas de aquisição e  modelo para armazenamento dos dados. Implementações de segurança foram apresentadas, como: criptografia, proteções contra tentativas de acesso mal intencionadas e níveis de controle de acesso dos usuários ao sistema. Por fim, foram apresentadas formas de utilização e consumo de serviços possíveis na plataforma.

Através deste modelo proposto, os módulos base para o funcionamento do sistema foram fielmente desenvolvidos. No Capítulo \ref{chap:interface-web}, foi introduzido o módulo da Interface de Gerenciamento do sistema, já batizado de RSCADA. Foram apresentados os serviços responsáveis pela disponibilidade da plataforma, assim como as linguagens utilizadas para a programação dela. As  etapas de uso do sistema desenvolvido foram demonstradas, como: (i) Autenticação ao Sistema, (ii) Cadastro de novos usuários, (iii) Criação de novos projetos e inclusões de objetos, variáveis e clientes à eles, (iv) Alarmes e notificações disponíveis para não-conformidades e (v) Ferramentas de Gerenciamento dos elementos já inseridos no sistema. Por fim, apresentada uma síntese das  funcionalidades do sistema com base na hierarquia dos usuários.

No Capítulo \ref{chap:resultados}, foram desenvolvidos quatro exemplos de projetos possíveis à serem implementados no sistema RSCADA, baseando-se em condições e necessidades de usos reais distintos. Os dois primeiros, utilizando microcontroladores diferentes e o protocolo \gls{MQTT}, ofereceram uma ideia sobre o uso na prática de pequenos dispositivos, que através deste protocolo podem realizar comunicação direta com o sistema \gls{SCADA}. Foram coletados dados, como: Temperatura, Umidade, Nível de Sinal e Latência de Conexão com total de quase 22 milhões de registros de informações nestes dois exemplos em poucos dias de captura, demonstrando a estabilidade e a capacidade computacional do sistema. Os dois últimos exemplos, possibilitaram a demonstração da capacidade de integração do sistema RSCADA com outros sistemas já existentes, onde nestes exemplos, foram integrados o \textit{software} MATLAB como intermediário de um processo e um conjunto de roteadores que havia um sistema proprietário.

A aplicação do sistema desenvolvido na prática, atendeu à todas as premissas iniciais sobre a facilidade de utilização e isenção de qualificação no gerenciamento de servidores que antes seria necessário. O principal potencial deste modelo é a facilidade de implementação de um novo projeto partindo do zero, sendo possível iniciar a aquisição de dados com novas plantas em minutos. Trabalhos Científicos focados na aquisição de dados em uma maior escala ou de forma remota e que, antes dependiam de conhecimento técnico acerca de servidores ou armazenamento destes dados, agora podem ser feitos em poucos cliques, apenas com um pequeno tratamento no envio destas informações à esta plataforma.

Apesar das desvantagens discutidas ao longo do trabalho sobre o sistema proposto, a gama de aplicações que se obtém um benefício pela implementação deste serviço em nuvem, é superior aos processos que tenham prejuízo pelo pequeno atraso no transporte das informações. Desta forma, deve ser feita uma ponderação no instante da escolha sobre qual sistema \gls{SCADA} será mais adequado ao processo estudado, sobre o quão necessário se faz a ausência do tempo de atraso ou a faixa de segurança que este tempo não causaria pertubações, além das implicações que a ausência de conexão possam causar ao processo.

\section{Plataforma Estudantil}
\label{sec:plataforma-estudantil}

Este trabalho tem como um de seus propósitos primordiais, a potencialização de trabalhos científicos desenvolvidos em meio acadêmico que, por muitas vezes houvessem um distanciamento entre o assunto trabalhado e a exigência de capacitação para o desenvolvimento de uma plataforma online, que faça coleta e análise de informações de várias grandes áreas da Engenharia Elétrica. Ou também, casos em que haja a necessidade de estudos de plantas industriais ou microcontroladores com sistemas \gls{SCADA}. Para isto, são disponibilizadas ferramentas completas se comparado ao uso de \textit{softwares} comerciais, de forma gratuita para estes estudantes. Com a plataforma online já desenvolvida e pronta para uso, é otimizado o desenvolvimento dos projetos que terão foco apenas no uso da integração do RSCADA.

Com o objetivo de ouvir e atender pedidos dos usuários sobre o desenvolvimento de novas funcionalidades ou a manutenção das já existentes, ferramentas de colaboração são integradas à interface de gerenciamento. Desta forma, é possível melhorar continuamente a qualidade de interação dos estudantes com a plataforma e possibilita que os resultados sejam atingidos de forma mais rápida.

\section{Trabalhos Futuros}
\label{sec:trabalhos-futuros}

Num contexto geral, a ideia de um sistema \gls{SCADA} estar inserido em um processo, é basicamente, sua execução em servidores locais com uma \gls{IHM} associada a estes. Quando se propõe a execução desta interface de gerenciamento diretamente pela \gls{WEB}, outros dispositivos úteis com maior mobilidade são inseridos, como: \textit{smartphones} e \textit{tablets}. Apesar da conexão à Internet nestes dispositivos serem substancialmente através de navegação \gls{WEB}, outras funcionalidades do sistema operacional poderiam ser aproveitadas pelo sistema desenvolvido neste trabalho caso houvesse um aplicativo nativo para eles. Estes dispositivos poderiam por exemplo, servir como um intermediário na comunicação com sensores que trabalham com \textit{bluetooth}, ou, auxiliar a configuração inicial de dispositivos projetados para trabalhar com redes Wi-Fi, possibilitando por exemplo, que o RSCADA pudesse identificar novos dispositivos na rede e gerar todos os parâmetros necessários para o funcionamento imediato.

Além do desenvolvimento destes aplicativos para plataformas nativas, uma possibilidade é o projeto de placas eletrônicas, que serviriam como \textit{drivers} de comunicação entre o RSCADA e dispositivos do processo que estejam legados ou que tenham difícil integração com os protocolos utilizados. Desta forma, haveria um ganho de retrocompatibilidade e a possibilidade de execução de algumas tarefas locais quando necessárias.

Por fim, outra possibilidade é o desenvolvimento de um módulo capaz de atender toda a plataforma de forma local que, na presença de conexão à internet, seja sincronizada com a estrutura em nuvem. Isto possibilitaria a mitigação do problema de atraso em transporte, já que se reduziria este tempo para algo em torno de 2 milisegundos e funcionaria como um intermediário para o pré tratamento dos dados.
	
	%Elementos pós-textuais	
	\bibliography{elementos-pos-textuais/referencias}
%	\imprimirglossario	
	\imprimirapendices
		% Adicione aqui os apendices do seu trabalho
        \input{elementos-pos-textuais/apendices/apendiceA}
		\apendice{Código: Qualidade de Sinal - Wi-Fi}
\label{ap:apendice-wifi}

\begin{lstlisting}[
    basicstyle=\tiny,
]
#include <WiFi.h>

const char* ssid = "# Cipriano";
const char* senha = "senha da minha casa";

#include <MQTT.h>

WiFiClient net;
MQTTClient mqtt;

const char* token = "438C1C";

String float2str(float x, byte precision = 2) {
  char tmp[50];
  dtostrf(x, 0, precision, tmp);
  return String(tmp);
}

bool conectaWiFi() {
  if (WiFi.status() != WL_CONNECTED) {
    delay(250);
    return 0;
  } else
    return 1;
}

void wdt() {
  yield();
}

void setup() {
  Serial.begin(9600);

  WiFi.disconnect(true);
  WiFi.mode(WIFI_STA);
  WiFi.begin(ssid, senha);
  WiFi.setSleep(false);
  
  conectaWiFi();
    
  mqtt.begin("mqtt.rscada.ga", net);
  mqtt.connect(token, token, token);
  mqtt.subscribe(String(token)+"/monitoramento");
}

bool monitoramento = true;

void callback(char* topico, byte* msg, unsigned int tamanho) {
    String mensagem;
  
	for (int i = 0; i < tamanho; i++)
		mensagem += (char)msg[i];

	if (String(topico) == String(token)+"/monitoramento")
		if(mensagem == "on")
			monitoramento = true;
		else
			monitoramento = false;
  	}
}

unsigned long tempoTotal = 0;
  
void loop() {
  if(conectaWiFi() && monitoramento == true){
    String sinal = String(WiFi.RSSI());

    unsigned long tempoInicial = millis();    

    if(mqtt.connected()){
      mqtt.publish(String(token)+"/sinal", sinal, false, 1);
      tempoTotal = millis() - tempoInicial;

      if(tempoTotal > 0)
        mqtt.publish(String(token)+"/latencia", String(tempoTotal), false, 1);
    } else {
      mqtt.disconnect();
      mqtt.connect(token, token, token);
    }

    while((millis() - tempoInicial) < 200) wdt();
  }
  wdt();
}
\end{lstlisting}
		\input{elementos-pos-textuais/apendices/apendiceC}
		\apendice{Código: Demanda de Roteadores}
\label{ap:apendice-roteadores}

\begin{lstlisting}[
    basicstyle=\tiny,
]
<?php
    $roteadores = array(
		"237" => "Recepcao",
		"238" => "Auditorio",
		"239" => "Apartamento 204",
		"240" => "Restaurante",
		"241" => "Apartamento 107",
		"242" => "Apartamento 216",
		"243" => "Apartamento 210",
		"244" => "Apartamento 226",
		"245" => "Apartamento 103",
		"246" => "Apartamento 213",
		"247" => "Apartamento 219",
	);

	$doisT = 0;
	$cincoT = 0;
	$downloadT = 0;
	$uploadT = 0;

	for($i = 237; $i < 248; $i++){
		$ip = "10.255.0.{$i}";

		$ch = curl_init();

		curl_setopt($ch, CURLOPT_COOKIEJAR, "cookie-{$ip}.txt");
		curl_setopt($ch, CURLOPT_URL,"http://{$ip}/Main_WStatus2g_Content.asp");
		curl_setopt($ch, CURLOPT_USERPWD, "rhulio:minhasenha");
		curl_setopt($ch, CURLOPT_HEADER, 0);
		curl_setopt($ch, CURLOPT_POST, 0);
		curl_setopt($ch, CURLOPT_RETURNTRANSFER, 1);
		curl_setopt($ch, CURLOPT_TIMEOUT, 5);

		$r = curl_exec($ch);
		preg_match_all(
		    "/[0-9A-F]{2}\:[0-9A-F]{2}\:[0-9A-F]{2}\:[0-9A-F]{2}\:[0-9A-F]{2}\:[0-9A-F]{2} /",
		$r, $mac);
		$dois = count($mac[0]);
		$doisT += $dois;

		curl_setopt($ch, CURLOPT_URL,"http://{$ip}/Main_WStatus_Content.asp");
		$r = curl_exec($ch);
		preg_match_all(
		    "/[0-9A-F]{2}\:[0-9A-F]{2}\:[0-9A-F]{2}\:[0-9A-F]{2}\:[0-9A-F]{2}\:[0-9A-F]{2} /",
		$r, $mac);
		$cinco = count($mac[0]);
		$cincoT += $cinco;

		curl_setopt($ch, CURLOPT_URL,"http://{$ip}/Main_TrafficMonitor_last24.asp");
		$r = curl_exec($ch);

		preg_match("/rx_total\: ([0-9]+)/", $r, $download);
		$download = number_format($download[1]/1073741824, 2, ".", "");
		$downloadT += $download;

		preg_match("/rx_max\: ([0-9]+)/", $r, $downloadS);
		$downloadS = number_format($downloadS[1]/125000, 2, ".", "");

		preg_match("/tx_total\: ([0-9]+)/", $r, $upload);
		$upload = number_format($upload[1]/1073741824, 2, ".", "");
		$uploadT += $upload;
		
		preg_match("/tx_max\: ([0-9]+)/", $r, $uploadS);
		$uploadS = number_format($uploadS[1]/125000, 2, ".", "");

		curl_close($ch);

		$totalAtivos = $dois+$cinco;
		$enviaAtivos = file_get_contents(
		    "http://sistema.rscada.ga/api/C487D2/envio?a{$i}={$totalAtivos}&d{$i}={$download}&u{$i}={$upload}&dl{$i}={$downloadS}&ul{$i}={$uploadS}"
		);

		echo "<strong>{$totalAtivos} ativos - {$dois} / {$cinco} - {$roteadores[$i]}</strong><br />Ultimas 24 horas:<br />{$uploadS} / {$downloadS} Mbps<br />{$upload} / {$download} GB<br /><br />";
\end{lstlisting}
	%\imprimiranexos
		% Adicione aqui os anexos do seu trabalho
		
	\imprimirindice

\end{document}